%%%%%%%%%%%%%%%%%%%%%%%%%%%%%%%%%
% Header and packages 
%%%%%%%%%%%%%%%%%%%%%%%%%%%%%%%%%

\documentclass[10pt]{article}
\usepackage[utf8]{inputenc}
\usepackage{amsfonts,amssymb,amsmath,amsthm}
\usepackage[a4paper, top=50pt, marginparwidth=25pt, textwidth=510pt, bottom=50pt]{geometry}
\usepackage[pdftex]{color,graphicx}
\usepackage{marginnote}
\usepackage[multiple]{footmisc}

\reversemarginpar
\theoremstyle{definition}
\newtheorem{thm}{Théorème}
\newtheorem{definition}{Définition}
\newtheorem{proposition}{Proposition}
\newtheorem{remark}{Remarque}
\newtheorem*{preuve}{Preuve}

%%%%%%%%%%%%%%%%%%%%%%%%%%%%%%%%%
% Symbols
%%%%%%%%%%%%%%%%%%%%%%%%%%%%%%%%%


\newcommand{\N}{\mathbb{N}}
\newcommand{\F}{\mathbb{F}}
\renewcommand{\L}{\mathbb{L}}
\newcommand{\Z}{\mathbb{Z}}
\newcommand{\Q}{\mathbb{Q}}
\newcommand{\D}{\mathbb{D}}
\newcommand{\R}{\mathbb{R}}
\newcommand{\C}{\mathbb{C}}
\newcommand{\K}{\mathbb{K}}
\renewcommand{\P}{\mathbb{P}}
\newcommand{\Pp}{$\mathcal{P}$}
\newcommand{\ch}{\textrm{ch}}
\newcommand{\sh}{\textrm{sh}}
\newcommand{\id}{\textrm{Id}}
\newcommand{\atan}{\textrm{Arctan}}
\newcommand{\acos}{\textrm{Arccos}}
\newcommand{\asin}{\textrm{Arcsin}}
\newcommand{\Mat}{\textrm{Mat}}
\newcommand{\Gl}{\textup{Gl}}
\newcommand{\Ker}{\textup{Ker}}
\newcommand{\E}{\textrm{E}}
\newcommand{\Gal}{\textrm{Gal}}
\newcommand{\Hom}{\textrm{Hom}}
\newcommand{\card}{\textup{card}}
\newcommand{\rg}{\textup{rg}}
\renewcommand{\Im}{\textup{Im}}
\renewcommand{\Gl}{\textrm{Gl}}

\begin{document}

%%%%%%%%%%%%%%%%%%%%%%%%%%%%%%%%%
% Title
%%%%%%%%%%%%%%%%%%%%%%%%%%%%%%%%%

\author{Pierre Chrétien}
\title{Sur les valeurs du polynôme de Deuring modulo p}
\date{Janvier 2026}
\maketitle

%%%%%%%%%%%%%%%%%%%%%%%%%%%%%%%%%
% Body
%%%%%%%%%%%%%%%%%%%%%%%%%%%%%%%%%
\section{Position du problème}

Soit $p \ge 3$ premier.  
Soit
\[
E_{\lambda} : y^2 = x(x-\lambda)(x-1), \qquad \lambda \in \F_p \setminus \{0,1\},
\]
une courbe elliptique sous forme de Legendre.
Il est bien connu (voir \cite{Silverman:EC}, Theorem V.4.1) que $E_{\lambda}$ est supersingulière si $\lambda$ est racine de
\[
H_p(x) = \sum_{i=0}^{m} \binom{m}{i}^2 x^i = 0,
\qquad \text{où } m = \frac{p-1}{2}.
\]

\noindent Néanmoins, la répartition des valeurs de $H_p(x)$, $x \in \F_p$, est remarquable de symétrie.
La raison principale est que $H_p(\lambda)$ est intimement lié à la trace du Frobenius de $E_{\lambda}/\F_p$ d'une part et que $|E_{\lambda}(\F_p)| \in 4 \Z$ (voir \cite{Auer-Top}).

\noindent Le but de ces notes personnelles est de clarifier en un seul document ces liens ainsi que d’expliquer certains arguments de \cite{Auer-Top}.
En particulier on propose des versions complètes des preuves de \cite{Auer-Top} les plus élémentaires possibles, ne faisant pas référence à la $2$-descente et en esquivant le plus possible la cohomologie galoisienne. 

\section{Comptage des points de $E$}

Nous suivrons \cite{Silverman:EC} IV.4.
Ici $q=p^r$, $p \ge 3$ et $r \ge 1$.
Soit $E/\F_q : y^2 = f(x) = x^3+ax^2+bx+c$, avec
\[
E(\F_q)
= \{(x,y)\in \F_q^2 \mid y^2 = f(x)\} \cup \{\mathcal{O}\}.
\]

\noindent Déterminons une expression de $|E(\F_q)|$.
Soit 
\begin{align*}
\chi : \; \F_q &\longrightarrow \{-1,0,1\}\\
y &\longmapsto
\begin{cases}
-1 & \text{si } y \notin (\F_q^\times)^2,\\
0  & \text{si } y = 0,\\
1  & \text{si } y \in (\F_q^\times)^2.
\end{cases}
\end{align*}

\noindent On en déduit
\[
|E(\F_q)| = \sum_{x\in \F_q} \bigl(1 + \chi(f(x))\bigr) + 1.
\]
\noindent En effet 
\begin{itemize}
  \item $f(x) \notin (\F_q^\times)^2 \iff y^2 = f(x)$ n’a pas de solution sur $\F_q$ ;
  \item $f(x) = 0 \iff y^2 = f(x)$ a une solution sur $\F_q$ ;
  \item $f(x) \in (\F_q^\times)^2 \iff y^2 = f(x)$ a deux solutions sur $\F_q$.
\end{itemize}
De plus  $\F_q^\times$ est cyclique et
\begin{align*}
\psi :  \; \F_q^\times & \longrightarrow (\F_q^\times)^2\\
 x & \longmapsto x^2
\end{align*}
a pour noyau  $\Ker(\psi) = \{x \in \F_q^\times \mid x^2 = 1\} = \{\pm 1\} $ ( car $ q > 2$ ).
Donc $\left|(\F_q^\times)^2\right| = \frac{q-1}{2}$ et $(\F_q^\times)^2$ est l'unique sous-groupe d'ordre $\frac{q-1}{2}$ de  $\F_q^\times = \langle \alpha \rangle$, donc $(\F_q^\times)^2 = \langle \alpha^2 \rangle$.

\begin{proposition}
$\forall  y \in \F_q, \chi(y) = y^{\frac{q-1}{2}}.$
\end{proposition}

\begin{preuve}
Tout d'abord, $\chi(0) = 0 = 0^{\frac{q-1}{2}}$.

\noindent Soit $y \in \F_q, \; \chi(y)=1 \iff y \in (\F_q^\times)^2
\iff y = (\alpha^2)^i,\ i \in \mathbb{Z}
\iff y^{\frac{q-1}{2}} = 1$.  
Détaillons cette dernière équivalence.
Soit $y \in \F_q$ tel que $ y = (\alpha^2)^i$, alors $y^{\frac{q-1}{2}} = \alpha^{(q-1)i} = 1$ car $\langle \alpha \rangle = \F_q^\times.$
Réciproquement,
\[
y^{\frac{q-1}{2}} = 1
\;\Rightarrow\;
o(y) \mid \frac{q-1}{2},
\]
donc $y$ est dans l’unique sous-groupe d’ordre $\frac{q-1}{2}$
du groupe cyclique $\F_q^\times$, c’est-à-dire $y \in (\F_q^\times)^2.$
\[\tag*{$\square$}\]
\end{preuve}

\noindent Ainsi,
\[
|E(\F_q)| = \sum_{x \in \F_q} \left(1 + \chi(f(x))\right) + 1
\qquad  \text{dans } \Z.
\]

\[
\boxed{
|E(\F_q)| = 1 + q + \sum_{x \in \F_q} f(x)^{\frac{q-1}{2}} \;\;\; \;\;\; (\dagger)
}
\qquad  \text{dans } \F_q.
\]


\begin{proposition}
\[
\sum_{x \in \F_q} x^i =
\begin{cases}
-1 & \text{si } q-1 \mid i,\\
0 & \text{si } q-1 \nmid i.
\end{cases}
\]
\end{proposition}

\begin{preuve}
Vu que $\F_q^\times$ est cyclique d’ordre $q-1$, il suffit
d’étudier les cas $0 \le i < q-1$.
Les polynômes de Newton sont
$p_k = \sum_{j=1}^{q-1} x_j^k \quad \text{dans } \F_q[x_1,\dots,x_{q-1}]$ et $\sigma_k$ les polynômes symétriques élémentaires dans $\F_q[x_1,\dots,x_{q-1}]$.
Les relations de Newton donnent (voir \cite{Fresnel-Anneaux})
\[
p_d = \sum_{k=1}^{d-1} (-1)^{k-1} \sigma_k p_{d-k}
+ (-1)^{d+1} d \sigma_d.
\]

\noindent Puisque $\F_q = \{x \in \overline{\F_p} \mid x^q - x = 0\}$,  $\F_q^\times = \{ x \in \F_q \mid x^{q-1} - 1 =0\rbrace$ et $X^{q-1}-1$ a pour fonctions symétriques élémentaires en ses racines valent $\sigma_1 = \cdots = \sigma_{q-2} = 0, \sigma_{q-1} = (-1)^{q-1}.$

\noindent Ici
\[
p_1 = \sum_{x \in \F_q} x = \sum_{x \in \F_q^\times} x = \sigma_1 = 0
\]
\[
p_2 = \sigma_1 p_1 - 2\sigma_2 = 0 - 2\sigma_2 = 0
\]
\[
\vdots
\]
\[
p_{q-2}
= \sum_{k=1}^{q-3} (-1)^{k-1} \sigma_k p_{q-2-k} + (-1)^{q-1}(q-2)\sigma_{q-2} = 0
\]

et
$ p_{q-1}
= \sum_{x \in \F_q} x^{q-1} = 0^{\,q-1} + \sum_{x \in \F_q^\times} x^{q-1} = 0 + \sum_{x \in \F_q^\times} 1
= q-1 = -1.$ 
\[\tag*{$\square$}\]
\end{preuve}
\vspace*{.6cm}


\noindent Donc dans la relation $(\dagger)$ seuls les monômes $f(x)^{\frac{q-1}{2}}$ de degré un multiple de $q-1$ contribuent à la somme.
Or \(\deg f(x)=3\), donc
\[
\deg f(x)^{\frac{q-1}{2}} = \frac{3}{2}(q-1)
\]
et les seuls multiples entiers de $q-1$ dans $[ 0;\frac32(q-1)]$ sont $0$ et $q-1$.
Néanmoins si on note $\alpha$ le coefficient constant de $f(x)^{\frac{q-1}{2}}$ alors $\sum_{x \in \F_q} \alpha = q\alpha = 0$ donc seul le monôme de degré $q-1$ contribue dans la somme $(\dagger)$. 
Soit \(A_q\) le coefficient de \(x^{q-1}\) dans \(f(x)^{\frac{q-1}{2}}\).
La discussion précédente fournit
\[
\boxed{|E(\F_q)| = 1 - A_q \qquad \text{dans } \F_q }
\]

\begin{proposition}
Pour tout \(r \in \mathbb{N}\), en notant \(A_{p^r}\) le coefficient de \(x^{p^r-1}\) dans\(f(x)^{\frac{p^r-1}{2}}\), on a 
\[
A_{p^{r+1}} = A_{p^r}\cdot A_p
\]
\end{proposition}

\begin{preuve}
Soit
\[
f(x) = x^3 + ax^2 + bx + c,
\qquad a,b,c \in \F_q.
\]

\[
(*)\qquad
f(x)^{\frac{p^{r+1}-1}{2}}
= f(x)^{\frac{p^{r+1}-p^r+p^r-1}{2}}
= f(x)^{\frac{p^r(p-1)}{2}} \cdot f(x)^{\frac{p^r-1}{2}}
= \bigl(f(x)^{\frac{p-1}{2}}\bigr)^{p^r}
\cdot f(x)^{\frac{p^r-1}{2}}
\]

\[
=
\left[
\sum_{i=0}^{\frac{3}{2}(p-1)} \alpha_i x^i
\right]^{p^r}
\cdot
\left[
\sum_{j=0}^{\frac{3}{2}(p^r-1)} \beta_j x^j
\right]
= \sum_{i=0}^{\frac{3}{2}(p-1)} \alpha_i^{p^r} x^{i p^r}
\cdot
\sum_{j=0}^{\frac{3}{2}(p^r-1)} \beta_j x^j
\]

\noindent dont tout monôme est de la forme
\[
\gamma_{ij} x^{i p^r + j},
\qquad
i \in \Bigl[\mkern-6mu\Bigl[0,\tfrac{3}{2}(p-1) \Bigr]\mkern-6mu\Bigr], \;  
j \in \Bigl[\mkern-6mu\Bigl[ 0,\frac{3}{2}(p^r-1)\Bigr]\mkern-6mu\Bigr]
\]

\noindent \'Etudions les solutions \(i,j\) d’une équation de la forme
\[
(**)\qquad p^{r+1}-1 = j + i p^r
\]
On a 
\[
(**) \Rightarrow
j = p^r(p-i)-1 \;\Rightarrow\; j \equiv -1 \; \mod p^r
\]
or
\[
0 \le j = k p^r -1 \le \tfrac{3}{2}(p^r-1)
\Rightarrow
2k p^r -2 \le 3p^r -3
\Rightarrow
1 \le p^r(3-2k)
\Rightarrow 3-2k \ge 0
\Rightarrow k \le \tfrac{3}{2}.
\]
De plus
\[
0 \le j = k p^r -1
\Rightarrow
k p^r \ge 1
\Rightarrow
k > 0
\]

\noindent Donc $k=1$, i.e. $j = p^r - 1$.
L’équation \((**)\) se lit alors
\[
p^{r+1}-1 = p^r -1 + i p^r
\Rightarrow
p^{r+1} = p^r(i+1)
\Rightarrow\quad i = p-1.
\]

\noindent En conclusion, dans le développement \((*)\), le coefficient \(A_{p^{r+1}}\) du monôme de degré \(p^{r+1}-1\) de \(f(x)^{\frac{p^{r+1}-1}{2}}\) ne provient que du produit des monômes de degré $p^r(p-1)$ (resp. $ p^r-1$) de $f(x)^{\frac{p^r(p-1)}{2}}$ (resp. $f(x)^{\frac{p^r-1}{2}}$).
On a donc
\[
A_{p^{r+1}} = A_{p^r}\cdot (A_p)^{p^r}.
\]
\[\tag*{$\square$}\]
\end{preuve}

\vspace*{.2cm}

\begin{proposition}
Pour \(q=p^n\)
\[
A_q = A_p^{\frac{q-1}{p-1}}.
\]
\end{proposition}
\section{Nombre de points rationnels d’une courbe sous forme de Legendre}

\subsection{Le polynôme de Deuring}

Nous nous restreignons désormais au cas \(q=p\) et
\[
E_\lambda/\F_p : \quad y^2 = x(x-1)(x-\lambda),
\qquad \lambda \in \F_p \setminus \{0,1\}.
\]

\noindent D’après le paragraphe précédent $ |E_\lambda(\F_p)| = 1 - A_p \mod p$ où \(A_p\) est le coefficient de \(x^{p-1}\) de $(x(x-1)(x-\lambda))^{\frac{p-1}{2}}$.
Posons $m = \frac{p-1}{2}$.
Le coefficient de \(x^{p-1}\) de $[x(x-1)(x-\lambda)]^m$ est le coefficient de \(x^m\) de $(x-1)^m(x-\lambda)^m$.
Or
\[
(x-1)^m(x-\lambda)^m
=
\sum_{i=0}^m \binom{m}{i}(-1)^i x^{m-i}
\cdot
\sum_{j=0}^m \binom{m}{j}(-\lambda)^j x^{m-j}.
\]

\noindent Les monômes de degré \(m\) de ce produit sont obtenus pour \(i\) et \(j\) tels que
\[
m-i + m-j = m
\;\Longleftrightarrow\;
m = i+j
\;\Longleftrightarrow\;
j = m-i.
\]

\noindent Le coefficient de \(x^m\) dans \((x-1)^m(x-\lambda)^m\) est donc
\[
A_p
=
\sum_{j=0}^m
\binom{m}{m-j}(-1)^{m-j}
\binom{m}{j}(-\lambda)^{j}
=
(-1)^m \sum_{j=0}^m \binom{m}{j}^2 \lambda^j.
\]

\noindent Donc
\[
\boxed{
|E_\lambda(\F_p)| = 1 - (-1)^m H_p(\lambda)
\mod p }
\]
où
\[
H_p(X) = \sum_{i=0}^m \binom{m}{i}^2 X^i,
\qquad m=\frac{p-1}{2}.
\]


\subsection{Cas $\lambda = 0$ et $\lambda = 1$}

La courbe $E_{\lambda}$ décrit une courbe elliptique si et seulement si $\lambda \notin \{0;1\}$. 
Ces deux valeurs sont donc traitées de manière indépendante dans ce paragraphe.
On a $\boxed{H_p(0) = 1}$.
Déterminons
\[
H_p(1) = \sum_{i=0}^m \binom{m}{i}^2 \quad \mod p
\]

\begin{proposition}
\[
\sum_{i=0}^m \binom{m}{i}^2
=
\binom{2m}{m}
=
(-1)^m \mod p
\]
\end{proposition}

\begin{preuve}
Soit $ m \in \mathbb{N}$.
\[
(1+x)^m(1+x)^m = (1+x)^{2m}
\Longleftrightarrow
\sum_{i=0}^m \binom{m}{i} x^i \cdot \sum_{j=0}^m \binom{m}{j} x^j
=
\sum_{k=0}^{2m} \binom{2m}{k} x^k.
\]

\noindent Le coefficient de \(x^m\) de chaque membre vaut
\[
\sum_{\substack{i+j=m \\ 0 \le i,j \le m}}
\binom{m}{i}\binom{m}{j}
=
\binom{2m}{m} \Longleftrightarrow
\sum_{i=0}^m \binom{m}{i}\binom{m}{m-i}
=
\binom{2m}{m}
\Longleftrightarrow
\sum_{i=0}^m \binom{m}{i}^2
=
\binom{2m}{m}.
\]

\noindent De plus
\[
\binom{p-1}{i} \equiv (-1)^i \quad \mod p,
\qquad 0 \le i \le p-1.
\]

\noindent En effet, $\binom{p-1}{0} = 1 \equiv (-1)^0  \mod p$  et $\binom{p-1}{p-1} = 1 \equiv (-1)^{p-1} \mod p$.
De plus, pour \(0 \le i \le p-2\),
\[
\binom{p-1}{i} + \binom{p-1}{i+1}
=
\binom{p}{i+1}
\equiv 0 \mod p
\]

\noindent D’où $\binom{p-1}{i+1} \equiv -\binom{p-1}{i} \mod p$ qui donne $\binom{p-1}{i} \equiv (-1)^i \mod p$.
\[\tag*{$\square$}\]
\end{preuve}


\begin{proposition}
\[\boxed{H_p(1) = (-1)^{\frac{p-1}{2}} \mod p }\]

\noindent On notera
\[ 
H_p(1) \equiv 1 \; \mod p
\iff m \textrm{ est pair}
\iff (-1)^{\frac{p-1}{2}} = 1
\iff -1 \in \F_p^2,
\]
\[ 
H_p(1) \equiv -1 \;\mod p
\iff m \textrm{ est impair}
\iff (-1)^{\frac{p-1}{2}} = -1
\iff -1 \notin \F_p^2.
\]
\end{proposition}

\section{Courbes de Legendre et isogénies}
On se limitera désormais au cas  \(p \ge 5\).
Le comportement remarquable de $\{ H_p(a),\ a \in \F_p \setminus \{0,1\} \}$ et plus précisément de $(H_p(2), H_p(3), \ldots, H_p(p-1))$ provient de \cite{Auer-Top} et des résultats intermédiaires qui y sont exposés.
On recopie le résultat principal de \cite{Auer-Top}. 

\vspace*{.4cm}
\begin{thm}
Soit \(E/\F_q\) une courbe elliptique.
On note \(q = r^2\), \(r \in \mathbb{N}\), si \(q\) est un carré.
\[
E \sim_{\F_q} E_\lambda, \quad \lambda \in \F_q
\quad \Longleftrightarrow \quad
|E(\F_q)| \in 4\mathbb{Z} \setminus \{(r+1)^2\}
\]
\end{thm}

\noindent D’après le théorème de Tate,
\[
E \sim_{\F_q} E_\lambda
\quad \Longleftrightarrow \quad
|E(\F_q)| = |E_\lambda(\F_q)|
\]

\noindent On a donc une description des ordres possibles pour \(|E_\lambda(\F_q)|\) quand \(\lambda\) varie. 
Nous allons nous contenter de suivre la preuve de \cite{Auer-Top} et d’expliciter les points délicats dans le cas qui nous intéresse, à savoir \(q=p\).
Nous allons donc prouver le corollaire suivant.


\begin{thm}
Soit \(E/\F_p\) une courbe elliptique.
\[
E \sim_{\F_p} E_\lambda, \quad \lambda \in \F_p
\quad \Longleftrightarrow \quad
|E(\F_p)| \in 4\mathbb{Z}
\]
\end{thm}

\subsection{Classes d’isomorphismes}

Il est bien connu (voir \cite{Silverman:EC} III 1.7 et sa preuve) 
\[
E_\lambda : y^2 = x(x-1)(x-\lambda)
\;\;\sim_{\overline{\F}_p}\;\;
E_\gamma : y^2 = x(x-1)(x-\gamma)
\]

\[
\Longleftrightarrow\quad
\gamma \in
\left\{
\lambda,\;
1-\lambda,\;
\frac{1}{\lambda},\;
1-\frac{1}{\lambda},\;
\frac{1}{1-\lambda},\;
1-\frac{1}{1-\lambda}
\right\}.
\]
Ce qui donne une description complète des classes d’isomorphisme des courbes sous forme de Legendre. 
Néanmoins :

\begin{enumerate}
\item ce n’est une description que sur $\overline{\F}_p$, donc à isomorphisme géométrique près,
\item la relation d’isomorphisme (même sur $\F_p$) n’est pas la bonne relation d’équivalence pour étudier $|E(\F_p)|$.
\end{enumerate}

\begin{thm}[Tate]
Soient $E_1, E_2$ des courbes elliptiques sur $\F_q$, $q =p^n$.
Il existe une isogénie $\F_q$-rationnelle $\varphi : E_1 \to E_2
\iff
|E_1(\F_q)| = |E_2(\F_q)|$
\end{thm}

\noindent La bonne relation d’équivalence à considérer pour le sujet qui nous intérese est donc $E_1 \sim_{\F_p} E_2$.
Nous avons cependant besoin de comprendre un certain nombre de relations provenant d’isomorphismes sur $\F_p$, nous collectons dans ce paragraphe des résultats sur ce sujet.
À partir de maintenant, $E_1 \simeq E_2:$ signifiera un isomorphisme $\F_p$-rationnel de courbes elliptiques $E_1/\F_p$, $E_2/\F_p$.
De même, $E_1 \sim E_2:$ signifiera une isogénie $\F_p$-rationnelle de courbes elliptiques $E_1/\F_p$, $E_2/\F_p$ et .
La courbe
\[
E_{\lambda} : y^2 = x(x-1)(x-\lambda), \qquad \lambda \in \F_p \setminus \{0;1\}
\]
a toute sa $2$-torsion $\F_p$-rationnelle 
\[
E_\lambda[2]
=
\{(0,0),(1,0),(\lambda,0),\mathcal{O}\}
\subset E_\lambda(\F_p).
\]

\begin{proposition}
Soit $E/\F_p : y^2 = x(x-\alpha)(x-\beta)$ une courbe elliptique.
\[
\exists \lambda \in \F_p \quad / \quad E \simeq E_\lambda 
\iff
\{ \pm \alpha,\ \pm \beta,\ \pm(\alpha-\beta)\} \cap \F_p^{\,2} \neq \emptyset
\]
\noindent Dans ce cas on dit que $E$ est \emph{Legendre isomorphe}.
\end{proposition}

\begin{preuve}Remarquons que $E/\F_p$ telle que $E[2]\subset E(\F_p)$
a un modèle $y^2 = (x-a)(x-b)(x-c), \quad a,b,c \in \F_p$
et l’automorphisme
\[
\begin{cases}
x \mapsto x+a \\
y \mapsto y
\end{cases}
\]
ramène au modèle $E : y^2 = x(x-\alpha)(x-\beta), \quad \alpha,\beta \in \F_p$.
On traite donc dans cet énoncé des courbes elliptiques sur $\F_p$ ayant leur $2$-torsion rationnelle.
On étudie donc les isomorphismes de courbes elliptiques suivants

\begin{align*}
 &\quad  y^2 = x(x-\alpha)(x-\beta) \;\simeq\;  y^2 = x(x-1)(x-\lambda)\\
\Longleftrightarrow & \quad  y^2 = x^3 - (\alpha+\beta)x^2 + \alpha\beta x \;\simeq\; y^2 = x^3 - (1+\lambda)x^2 + \lambda x
\end{align*}

\noindent Or l'expression d'un isomorphisme de courbes elliptiques sous forme de Weiestrass est bien connue.
Par exemple \cite{Silverman:EC} III, Table 3. fournit 
\[
\Longleftrightarrow (S) :
\begin{cases}
-u^2(\alpha+\beta) = -(1+\lambda) + 3R - \lambda s^2, \\
u^3\cdot 0  = 2t, \\
u^4 \alpha\beta = \lambda - 2R(1+\lambda) + 3R^2 - 2st, \\
u^6\cdot 0 =  R\lambda - R^2(1+\lambda) + R^3 - t^2.
\end{cases}
 \Longleftrightarrow (S) :
\begin{cases}
- u^2(\alpha+\beta) = -(1+\lambda) + 3R, \\
u^4 \alpha\beta = \lambda - 2R(1+\lambda) + 3R^2, \\
0 = R(\lambda - (1+\lambda)R + R^2).
\end{cases}
\]

\noindent Or $0 = R(\lambda - (1+\lambda)R + R^2) \;\Longleftrightarrow\; R \in \{0,1,\lambda\}.$

\begin{itemize}
\item Si $R=0$ : 
\[(S)\Longleftrightarrow
\begin{cases}
u^2(\alpha+\beta) = 1+\lambda, \\
u^4\alpha\beta = \lambda,
\end{cases}
\quad\Longleftrightarrow\quad
\begin{cases}
\alpha+\beta = \dfrac{1}{u^2} + \dfrac{\lambda}{u^2}, \\
\alpha\beta = \dfrac{1}{u^2}\cdot\dfrac{\lambda}{u^2}.
\end{cases}
\Longleftrightarrow
\{\alpha,\beta\} = \left\{\dfrac{1}{u^2},\dfrac{\lambda}{u^2}\right\}.
\]


\item Si $R=\lambda$ : 
\[(S)\Longleftrightarrow
\begin{cases}
u^2(\alpha+\beta) = 1-2\lambda, \\
u^4\alpha\beta = \lambda^2-\lambda,
\end{cases}
\quad\Longleftrightarrow\quad
\begin{cases}
\alpha+\beta = \dfrac{-\lambda}{u^2} + \dfrac{1-\lambda}{u^2}, \\
\alpha\beta = \dfrac{-\lambda}{u^2}\cdot\dfrac{1-\lambda}{u^2}.
\end{cases}
\Longleftrightarrow
\{\alpha,\beta\} =
\left\{\dfrac{-\lambda}{u^2},\dfrac{1-\lambda}{u^2}\right\}.
\]
\noindent Auquel cas $\alpha - \beta$ ou $\beta-\alpha$ vaut $\dfrac{1-\lambda}{u^2} - \dfrac{-\lambda}{u^2} = \dfrac{1}{u^2}$.

\item Si $R=1$ : 
\[ (S)\Longleftrightarrow
\begin{cases}
u^2(\alpha+\beta) = \lambda - 2, \\
u^4\alpha\beta = 1-\lambda,
\end{cases}
\quad\Longleftrightarrow\quad
\begin{cases}
\alpha+\beta = \dfrac{-1}{u^2} + \dfrac{\lambda-1}{u^2}, \\
\alpha\beta = \dfrac{1}{u^2}\cdot\dfrac{\lambda-1}{u^2}.
\end{cases}
\Longleftrightarrow
\{\alpha,\beta\}
=
\left\{\dfrac{-1}{u^2},\dfrac{\lambda-1}{u^2}\right\}.
\]
\noindent Auquel cas $- \alpha$ ou $- \beta$ vaut $\dfrac{1}{u^2}$.
\end{itemize}
\[\tag*{$\square$}\]
\end{preuve}


\begin{proposition}\label{2descente}
Soit $E/\F_p : y^2 = (x-\alpha)(x-\beta)(x-\gamma)$
une courbe elliptique. Alors
\[
(\gamma,0)\in 2E(\F_p)
\iff
\gamma-\alpha,\ \gamma-\beta \in (\F_p^\times)^2
\]
\end{proposition}

\begin{preuve}
Il s’agit de caractériser l’image de la multiplication par $2$ de $E(\F_p)$.
\cite{Auer-Top} renvoit à \cite{Silverman:EC} \S X.1 qui traite de la $2$-descente (dans le cas des corps de nombres donc) ou à \cite{Schaefer} qui n’énonce un résultat que pour les courbes hyperelliptiques.
L’appendice A à la fin de cette note propose un exposé minimal des notions nécessaires, il s'agit encore de résultats bien connus.
\[\tag*{$\square$}\]
\end{preuve}


\subsection{Twists quadratiques}

\begin{definition}
Soit $E/k : y^2 = f(x)$ une courbe elliptique sous forme de Weierstrass.
Soit $\alpha \in k^\times$ alors
\[
E^{(\alpha)} : \alpha y^2 = f(x)
\]
est une courbe elliptique appelée \emph{twist quadratique} de $E$ par $\alpha$.
\end{definition}

\begin{remark}
\begin{enumerate}
\item
$\alpha y^2 = f(x) = x^3 + a x^2 + b x + c
\simeq
y^2 = x^3 + \alpha a x^2 + \alpha^2 b x + \alpha^3 c$
par
\[
\varphi(x,y) = \left(\dfrac{x}{\alpha},\dfrac{y}{\alpha^2}\right)
\]
Notons que $\varphi$ est un isomorphisme de courbes algébriques mais pas un isomorphisme de courbes elliptiques.
\item Le twist est trivial si eu seulement si $E \simeq_k E^{(\alpha)}$.
\item Soit $E/\F_q$ et $\alpha \notin (\F_q^\times)^2$ alors $|E(\F_q)| + |E^{(\alpha)}(\F_q)| = q + 2$.
\end{enumerate}
\end{remark}

\begin{proposition}\label{j1728}
Soit $E/k$ une courbe elliptique et $\alpha \notin (k^\times)^2$.
\[
E \simeq E^{(\alpha)} \;\Longrightarrow\;
j(E)=1728 \quad\text{et}\quad k(\alpha)=k(\sqrt{-1})
\]
\end{proposition}

\begin{preuve}
\noindent On suppose ici que $k=p\ge 5$ bien que le résultat reste vrai pour $p=3$.
Cette restriction permet de donner un modèle de $E/k$ sous forme de Weierstrass courte, ce qui simplifie l’expression de $j(E)$ et donc la preuve ci-dessous. 
Les formules pour $p=3$ sont disponibles dans\cite{Silverman:EC} Appendix A.

\noindent Soit
\[
E/k : y^2 = x^3 + Ax + B
\quad\text{alors}\quad
j(E) = 1728\,\frac{4A^3}{4A^3+27B^2}.
\]


\noindent Si $E \simeq E^{(\alpha)}$, alors $E^{(\alpha)}$ a pour modèles $ y^2 = x^3 + \alpha^2 A x + \alpha^3 B$.
Un isomorphisme $\varphi : E \longrightarrow E^{(\alpha)}$ est de la forme $\varphi(x,y) = (u^2 x, u^3 y)$ et vérifie
\[
\begin{cases}
\dfrac{A}{u^4} = \alpha^2 A, \\[10pt]
\dfrac{B}{u^6} = \alpha^3 B.
\end{cases}
\]
\begin{itemize}

\item Si $A\neq 0$ et $B\neq 0$, alors
\[
\begin{cases}
\alpha^2 = \dfrac{1}{u^4}, \\[10pt]
\alpha^3 = \dfrac{1}{u^6},
\end{cases}
\quad\Longrightarrow\quad
\alpha = \left(\dfrac{1}{u}\right)^2 \in (k^\times)^2.
\]
ce qui est absurde.


\item Si $A=0$, alors $B\neq 0$ sinon $E$ n’est pas lisse en $(0,0)$.
On a
\[
\alpha^3 = \dfrac{1}{u^6}
\quad\Longrightarrow\quad
\alpha = \dfrac{1}{\alpha^2}\cdot\dfrac{1}{u^6}
= \left(\dfrac{1}{\alpha u^3}\right)^2 \in (k^\times)^2.
\]
ce qui est absurde.

\item Donc $B = 0$, d'où $j(E) = 1728$.
De plus $A \neq 0$, d’où $ \alpha^2 = \frac{1}{u^4} $ donc $ \alpha \in \{ \frac{1}{u^2};  -\frac{1}{u^2} \}$.
Or  $\alpha \notin (k^\times)^2 \text{ donc } \alpha = -\frac{1}{u^2}$.
En particulier  $k(\sqrt{\alpha}) = k(\sqrt{-1}).$
\end{itemize}
\[\tag*{$\square$}\]
\end{preuve}


\begin{proposition}\label{iso-legendre}
 $\forall \lambda \in \F_q \setminus \{ 0; 1 \}$
\[E_\lambda^{(-1)} \simeq E_{1-\lambda}
\qquad
E_\lambda^{(\lambda)} \simeq E_{1/\lambda}
\qquad
E_\lambda^{(1-\lambda)} \simeq E_{\lambda/(\lambda-1)}
\]
\end{proposition}

\begin{preuve}
\begin{enumerate}
\item $E_\lambda : y^2 = x(x-1)(x-\lambda) = x^3 - (1+\lambda)x^2 + \lambda x$.
Donc
\[
E_\lambda^{(-1)} :
y^2 = x^3 + (1+\lambda)x^2 + \lambda x
= x(x+1)(x+\lambda).
\]

\noindent Et $\varphi(x,y) = (x-1,\, y)$ fournit un isomorphisme $E_\lambda^{(-1)} \simeq E_{1-\lambda}$.
\item $E_\lambda^{(\lambda)} : y^2 = x^3 - (1+\lambda)\lambda x^2 + \lambda^3 x= x(x-\lambda)(x-\lambda^2)$ et $E_{1/\lambda} : 
y^2 = x(x-1)\left(x-\frac{1}{\lambda}\right)$.
Donc
$\varphi(x,y) = (\lambda^2 x,\, \lambda^3 y)$ fournit $E_\lambda^{(\lambda)} \simeq E_{1/\lambda}$.

\item $E_\lambda^{(1-\lambda)} : y^2 = x^3 - (1+\lambda)(1-\lambda)x^2 + \lambda(1-\lambda)^2 x= x(x-(1-\lambda))(x-\lambda(1-\lambda))$.

On vérifie que $\varphi(x,y) = \bigl((1-\lambda)^2 x + \lambda(1-\lambda),\; (1-\lambda)^3 y \bigr)$ fournit $E_\lambda^{(1-\lambda)} \simeq E_{\lambda/(\lambda-1)}$.
En effet 
\begin{align*}
&\varphi(y^2) = \varphi(x)\,\varphi(x-(1-\lambda))\,\varphi(x-\lambda(1-\lambda))\\
\Leftrightarrow \quad &
(1-\lambda)^6 y^2 = \bigl[(1-\lambda)^2 x + \lambda(1-\lambda)\bigr]\bigl[(1-\lambda)^2 x + \lambda(1-\lambda)-(1-\lambda)\bigr] \bigl[(1-\lambda)^2 x\bigr]\\
\Leftrightarrow\quad &
y^2 =\left(x+\frac{\lambda}{1-\lambda}\right)\left(x+\frac{\lambda}{1-\lambda}-\frac{1}{1-\lambda}\right)x
= x(x-1)\left(x-\frac{\lambda}{\lambda-1}\right).
\end{align*}
\end{enumerate}
\[\tag*{$\square$}\]
\end{preuve}


\begin{proposition}\label{carac-legendre}
Soit $\lambda \in \F_q \setminus \{0,1,-1,2,\tfrac12\}$.
Les assertions suivantes sont équivalentes :
\begin{enumerate}
\item[(a)]
\(
E_\lambda \simeq E_\mu
\)
pour tout $\mu \in \left\{\lambda,\; 1-\lambda,\;\frac{1}{\lambda},\;1-\frac{1}{\lambda},\;\frac{1}{1-\lambda},\;\frac{\lambda}{\lambda-1} \right\}$.
\item[(b)]
\(
-1,\ \lambda,\ 1-\lambda \in (\F_q^\times)^2.
\)

\item[(c)]
\(
E_\lambda[4](\F_q)
\simeq \mathbb{Z}/4\mathbb{Z} \times \mathbb{Z}/4\mathbb{Z}.
\)
\end{enumerate}
Si \(E_\lambda^{(\alpha)}/\F_q\) n’est pas Legendre isomorphe pour un \(\alpha \in \F_q^\times\), alors les assertions précédentes sont satisfaites.
\end{proposition}

\begin{preuve}
L’hypothèse sur \( \lambda \) assure que $-1 \notin \left\{ \lambda,\ldots,\frac{\lambda}{\lambda-1}\right\}$, donc $j(E_\lambda) \neq 1728$, (voir \cite{Silverman:EC} III 1.7).
\begin{enumerate}
\item[(a) $\Rightarrow$ (b)]
Supposons que $-1$ ou $\lambda$ ou $1-\lambda \notin (\F_q^\times)^2$.
D’après la proposition \ref{j1728}, comme $j(E_\lambda) \neq 1728$ 
\[
E_\lambda^{(-1)} \not\simeq E_\lambda
\ \text{ou}\
E_\lambda^{(\lambda)} \not\simeq E_\lambda
\ \text{ou}\
E_\lambda^{(1-\lambda)} \not\simeq E_\lambda.
\]
\noindent Ce qui, avec la Proposition \ref{iso-legendre} contredit (a). 
Donc $-1,\ \lambda,\ 1-\lambda \in (\F_q^\times)^2$.

\item[(b) $\Rightarrow$ (a)]
Si $\alpha \in (k^\times)^2$, alors $E^{(\alpha)} \simeq_k E$.
Donc si $-1,\ \lambda,\ 1-\lambda \in (\F_q^\times)^2$, alors
\[
E_\lambda^{(-1)} \simeq E_\lambda,
\qquad
E_\lambda^{(\lambda)} \simeq E_\lambda,
\qquad
E_\lambda^{(1-\lambda)} \simeq E_\lambda.
\]

\noindent Or
\[
E_\lambda^{(-1)} \simeq E_{1-\lambda},
\qquad
E_\lambda^{(\lambda)} \simeq E_{1/\lambda},
\qquad
E_\lambda^{(1-\lambda)} \simeq E_{\lambda/(\lambda-1)}.
\]

\noindent De plus $\frac{\lambda}{\lambda-1} = \frac{-\lambda}{1-\lambda} \in (\F_q^\times)^2$, donc
\[
E_{\lambda/(\lambda-1)}
\simeq
E_{\lambda/(\lambda-1)}^{\lambda/ (\lambda-1)}
\simeq
E_{(\lambda-1)/\lambda}
\simeq
E_{1-1/\lambda}.
\]
\noindent Enfin, $\ 1-\lambda \in (\F_q^\times)^2 \Rightarrow E_{1-\lambda}^{(1-\lambda)} \simeq E_{1-\lambda}$.
Or $E_{1-\lambda}^{(1-\lambda)} \simeq E_{1/(1-\lambda)}$.
\item[(b) $\Rightarrow$ (c)]
On a
\[
0-1,\; 0-\lambda,\; 1-\lambda,\; 1-0,\; \lambda-0,\; \lambda-1
\in (\F_q^\times)^2.
\]

D'après la Proposition \ref{2descente}, $(0,0),\ (1,0),\ (\lambda,0) \in [2]E_\lambda(\F_q)$.
Par exemple,
\[
(0,0) = 2P_0,
\qquad
P_0 \in E_\lambda(\F_q),
\]
et $-P_0 \in E_\lambda(\F_q)$.
Or $P_0 \neq -P_0$, sinon $2P_0 = \mathcal{O}$, ce qui est absurde.
Donc, $P_0 \in E_\lambda[4](\F_q)$ et $P_0, -P_0, 2P_0$ sont trois éléments distincts de $ E_\lambda[4](\F_q)$.
De même avec les points suivants
\[
(1,0) = 2P_1, \; (\lambda,0) = 2P_\lambda, \qquad P_1, P_\lambda \in E_\lambda[4](\F_q).
\]

Donc $|E_\lambda[4](\F_q| > 10)$ et est un sous-groupe de $E_\lambda[4](\overline{\F}_q) \simeq \mathbb{Z}/4\mathbb{Z} \times \mathbb{Z}/4\mathbb{Z}$.
Par conséquent
\[
E_\lambda[4](\F_q)
\simeq
\mathbb{Z}/4\mathbb{Z} \times \mathbb{Z}/4\mathbb{Z}.
\]

\item[(c) $\Rightarrow$ (b)]
On a $E_\lambda[4](\F_q) = E_\lambda[4](\overline{\F}_q)$.
De plus $[2] : E_\lambda(\overline{\F}_q) \rightarrow E_\lambda(\overline{\F}_q)$ est surjective. 
Ainsi,
\[
\forall P \in E_\lambda[2](\overline{\F}_q), \quad
\exists Q \in E_\lambda(\overline{\F}_q) \; / \; 2Q = P
\]

\noindent Or $2P = \mathcal{O}$, d'où $ Q \in E_\lambda[4](\overline{\F}_q) = E_\lambda[4](\F_q)$ i.e. toute la \(2\)-torsion est dans \(2E_\lambda(\F_q)\).
La proposition \ref{2descente} assure
\[
0-1,\; 0-\lambda,\; 1-0,\; 1-\lambda,\; \lambda-0,\; \lambda-1
\in (\F_q^\times)^2,
\]
i.e. \[
-1,\ \lambda,\ 1-\lambda \in (\F_q^\times)^2.\]
\end{enumerate}


\noindent Nous avons prouvé que les assertions (a), (b) et (c) sont équivalentes.
Si \(\alpha \in (\F_q^\times)^2,\) alors \( E_\lambda^{(\alpha)} \simeq E_\lambda \) donc \(E_\lambda^{(\alpha)}\) est Legendre isomorphe.
Supposons donc que \(E_\lambda^{(\alpha)}\) ne soit pas Legendre isomorphe, en particulier $\alpha \notin (\F_q^\times)^2$.
Supposons par l'absurde que  $-1 \notin (\F_q^\times)^2$.
\[
\left| \F_q^\times / (\F_q^\times)^2 \right| = 2 \; \Rightarrow  \; \frac{\alpha}{-1} \in (\F_q^\times)^2,
\]

\noindent En notant $\delta^2 = -\alpha, \; \delta \in \F_q^\times$, on a un isomorphisme 
\begin{align*}
\varphi : E_\lambda^{(-1)}& \xrightarrow{\sim} E_\lambda^{(\alpha)}\\
 (x,y) &\mapsto \left(\frac{x}{\delta^2}, \frac{y}{\delta^3}\right),
\end{align*}

\noindent On en déduit $E_\lambda^{(\alpha)} \simeq   E_\lambda^{(-1)} \simeq E_{1-\lambda}$, ce qui est absurde car $E_\lambda^{(\alpha)}$ n'est pas Legendre isomorphe.
Donc $-1 \in (\F_q^\times)^2$.
De même $\lambda, 1-\lambda \in (\F_q^\times)^2$.
\[\tag*{$\square$}\]
\end{preuve}


\subsection{Preuve du résultat principal}

\begin{preuve}
Supposons que $E/\F_p \sim E_\lambda/\F_p$,  avec $\lambda \in \F_p$.
Alors $|E(\F_p)| = |E_\lambda(\F_p)|$.
Or $E_\lambda[2](\F_p) \simeq \mathbb{Z}/2\mathbb{Z} \times \mathbb{Z}/2\mathbb{Z}$, donc
\[
|E(\F_p)| \in 4\Z
\]

\noindent Inversement, supposons que $|E(\F_p)| \in 4\mathbb{Z}$.
Si $E[2](\F_p) \neq E[2](\overline{\F}_p)$, alors \(E[2](\F_p)\) ne contient pas de sous-groupe isomorphe à \(\mathbb{Z}/2\mathbb{Z} \times \mathbb{Z}/2\mathbb{Z}\).
Or
\[
E(\F_p) \simeq \mathbb{Z}/n_1\mathbb{Z} \times \mathbb{Z}/n_2\mathbb{Z},
\quad n_1 \mid n_2.
\]
Si $2 \mid n_1$, alors $2 \mid n_2$ et $\mathbb{Z}/2\mathbb{Z} \times \mathbb{Z}/2\mathbb{Z} \subset E(\F_p)$, c'est absurde, d'où $ 2 \nmid n_1$ et $4 \mid n_2$.
Ainsi, $E(\F_p)$ contient un élément \(P\) d’ordre \(4\), en particulier $2P \in E[2](\F_p) \subsetneq E[2](\overline{\F}_p)$.
Soit $Q \in E[2](\overline{\F}_p) \setminus E[2](\F_p)$ et \[
\varphi : E \longrightarrow E/\langle 2P \rangle = \widetilde{E}
\]
l'isogénie quotient.
Notons pour plus tard que $\varphi$ et \(\widetilde{E}\) sont définies sur \(\F_p\).
Montrons que $\widetilde{E}[2](\F_p) \simeq \mathbb{Z}/2\mathbb{Z} \times \mathbb{Z}/2\mathbb{Z}$.

\begin{itemize}
\item D'abord 
\[
\begin{cases}
 2\varphi(P) = \varphi(2P) = \mathcal{O} \\
\varphi(P) \neq \mathcal{O} \quad\text{car}\quad P \notin \langle 2P \rangle.
\end{cases}\quad\Rightarrow\quad \circ(\varphi(P)) = 2.
\]

\item
Ensuite 
\[ 
\begin{cases}
2\varphi(Q) = \varphi(2Q) = \varphi(\mathcal{O}) = \mathcal{O} \quad \text{car } Q \in  E[2]\\
\varphi(Q) \neq \mathcal{O}
\quad \text{car } Q \notin \langle 2P \rangle.
\end{cases}\quad \Rightarrow \quad \circ(\varphi(Q)) = 2.
\]


\item Soit \( n \in \mathbb{N} \) tel que $E[2](\F_{p^n}) = E[2](\overline{\F}_p)$, et soit $ \sigma$ un générateur de $ \Gal(\F_{p^n}/\F_p)$.
Puisque $Q \notin \langle 2P \rangle$ on a \[E[2](\F_{p^n}) = \langle 2P, Q \rangle\]
De plus 
\[
\sigma(Q) = Q + 2P,
\]
(si \( \sigma(Q) = Q \), alors \( Q \in E(\F_p) \), et si \( \sigma(Q) = 2P \), alors \( Q = \sigma^{-1}(2P) \in E(\F_p) \)).

D’où
\begin{align*}
\sigma(\varphi(Q))& = \varphi(\sigma(Q)) &\quad \text{car } \varphi \text{ est } \F_p\text{-rationnelle}\\
&= \varphi(Q + 2P)
= \varphi(Q) + \varphi(2P) = \varphi(Q) + \mathcal{O}= \varphi(Q)
\end{align*}

Donc $\varphi(Q) \in \widetilde{E}(\F_p)$.
Par conséquent,
\[
\widetilde{E}[2](\F_p)
= \langle \varphi(P), \varphi(Q) \rangle
\simeq
\mathbb{Z}/2\mathbb{Z} \times \mathbb{Z}/2\mathbb{Z}.
\]

En effet,
\[
\langle \varphi(P) \rangle =  \langle \varphi(Q) \rangle
\;\Longleftrightarrow\;
\varphi(P) = \varphi(Q)
\;\Longleftrightarrow\;
P - Q \in \langle 2P \rangle
\Longleftrightarrow\;
Q = P \ \text{ou}\ Q = 3P,
\]
ce qui est absurde.
\end{itemize}


\noindent Ainsi, l’isogénie $\varphi : E \longrightarrow \widetilde{E}$ permet de supposer qu'à \(\F_p\)-isogénie près
\[
E[2](\F_p)
\simeq
\mathbb{Z}/2\mathbb{Z} \times \mathbb{Z}/2\mathbb{Z}.
\]
\noindent On peut écrire un modèle $E: y^2 = (x-a)(x-b)(x-c),
\, a,b,c \in \F_p $ puis  appliquer l’automorphisme $(x,y) \longmapsto (x+a,\, y)$ pour obtenir le modèle
\[
E/\F_p : \quad
y^2 = x(x-\alpha)(x-\beta),
\qquad \alpha,\beta \in \F_p.
\]

\noindent De plus,
\begin{align*}
E^{(\alpha)} : \quad
y^2 = x^3 - \alpha(\alpha+\beta)x^2 + \alpha^2\beta x \;&\simeq\; E_{\beta/\alpha} :\;
y^2 = x^3 - \left(1+\frac{\beta}{\alpha}\right)x^2 + \frac{\beta}{\alpha}x\\
(x,y)& \longmapsto (\alpha^2 x,\ \alpha^3 y).
\end{align*}

\noindent Donc $ E^{(\alpha)} \simeq E_\lambda, \; \lambda = \frac{\beta}{\alpha} \in \F_p$.
Si $\alpha \in (\F_p^{\times})^2$ alors $E \simeq E^{(\alpha)} \simeq E_\lambda$ et la preuve est complète.
On suppose à partir de maintenant que  $\alpha \notin (\F_p^{\times})^2$.
On examine les cas supersinguliers et ordinaires séparément.
\begin{itemize}
\item
Si \(E\) est supersingulière, alors
\[
|E(\F_p)| = p + 1 - t
\quad \text{et} \quad
p \mid t.
\]
Or
\[
\begin{cases}
|t| \le 2\sqrt{p}\\
t = kp,
\end{cases}
\Rightarrow
k^2 p^2 \le 4p \;\Rightarrow\; k^2 p \le 4.
\]

Si \(p \ge 5\), alors \(k = 0\), i.e. $|E(\F_p)| = p + 1$.
Si \(p \le 3\), alors \(k \in \{-1,0,1\}\), cependant
\[
\begin{cases}
k = 1 \Rightarrow t = p \Rightarrow |E(\F_p)| = 1 \notin 4\mathbb{Z},\\
k = -1 \Rightarrow |E(\F_p)| = 2p + 1 \notin 4\mathbb{Z}.
\end{cases}
\]

Donc $k = 0$ i.e. $|E(\F_p)| = p + 1$.
Or $\alpha \notin (\F_p^{\times})^2$, d'où $ |E^{(\alpha)}(\F_p)| + |E(\F_p)| = 2p + 2$, i.e.
\[
|E^{(\alpha)}(\F_p)| = p + 1 = |E(\F_p)|.
\]

D’après le théorème de Tate, $E \sim E^{(\alpha)}$.
On a donc $E \sim E^{(\alpha)} \sim E_\lambda$,
i.e. \(E\) est Legendre isogène sur \(\F_p\).

\item
Si \(E\) est ordinaire.
On distingue deux cas

\begin{enumerate}
\item[(a)] Si \(E_\lambda^{(\alpha)}\) est isogène de Legendre, c'-à-d $E_\lambda^{(\alpha)} \sim E_\mu, \, \mu \in \F_p$.
En notant \(\operatorname{Tr}_{\F_p}(E)\) la trace du Frobenius de \(E/\F_p\), si
\(
E \not\sim E^{(\alpha)}
\),
alors
\[
\operatorname{Tr}_{\F_p}(E)
= -\operatorname{Tr}_{\F_p}(E^{(\alpha)})
= -\operatorname{Tr}_{\F_p}(E_\lambda)
= +\operatorname{Tr}_{\F_p}(E_\lambda^{(\alpha)}),
\]
car $E^{(\alpha)} \simeq E_\lambda$ et $ \alpha \notin (\F_p^\times)^2$.
Donc, d'après le théorème de Tate $ E \sim E_\lambda^{(\alpha)}$ d'où $E \sim E_\lambda^{(\alpha)} \sim E_\mu$.
Si au contraire $E \sim E^{(\alpha)}$, alors $E \sim E^{(\alpha)} \sim E_\lambda$. 
Dans tous les cas \(E\) est Legendre isogène sur \(\F_p\).
\item[(b)]Si \(E_\lambda^{(\alpha)}\) n’est pas  Legendre isogène, alors la proposition \ref{carac-legendre}  assure que
\[
\begin{cases}
E_\lambda[4](\F_p) \simeq \mathbb{Z}/4\mathbb{Z} \times \mathbb{Z}/4\mathbb{Z},\\
-1 \in (\F_p^\times)^2.
\end{cases}
\]

De plus,
\[
|E_\lambda(\F_p)|
= 2^\alpha \cdot \prod_{p_i \text{ premier } \geq 3} p_i^{m_i},
\qquad
\alpha \ge 4,
\]
Alors le théorème de Rück \cite{Ruck} donne l’existence d’une courbe $E'/\F_p$ telle que
\[
\begin{cases}
E_\lambda \sim E', \\
E'[4](\F_p) \simeq \mathbb{Z}/4\mathbb{Z} \times \mathbb{Z}/2\mathbb{Z}.
\end{cases}
\]

En effet, \(a = 1\) vérifie $ 0 \le a \le \min\bigl(v_2(p-1), \lfloor \alpha/2 \rfloor \bigr)$, d’où l’existence de \(E'\) telle que
\[
E'[2^a](\F_p) \simeq \mathbb{Z}/2\mathbb{Z} \times \mathbb{Z}/2^{\alpha-1}\mathbb{Z}.
\]
Donc
\[
E'[4](\F_p)
\simeq
\mathbb{Z}/2\mathbb{Z} \times \mathbb{Z}/4\mathbb{Z}.
\]

Soit $P \in E'(\F_p)$ d’ordre \(4\), et $Q = 2P$ , i.e. $Q \in E'[2](\F_p)$.
Quitte à effectuer $(x,y) \mapsto (x+s, y)$, on peut supposer qu'on a un modèle  \(E' : y^2 = x(x-\alpha')(x-\beta'), \; \alpha', \beta' \in \F_p\),  avec $(0,0) \in 2E'(\F_p)$.
La proposition \ref{2descente} assure alors 
\[
0-\alpha' \in (\F_p^\times)^2.
\]

Donc
\[
\frac{-\alpha'}{-1} =  \alpha' \in (\F_p^\times)^2,
\quad \text{i.e.} \quad
\alpha' = \delta^2,
\ \delta \in \F_p.
\]

On a alors avec $\lambda' = \frac{\beta'}{\alpha'}$
\begin{align*}
E' : \ y^2 = x^3 - (\alpha' + \beta')x^2 + \alpha'\beta' x &\;\simeq\; E_{\lambda'} :\;
y^2 = x^3 - \left(1+\frac{\beta'}{\alpha'}\right)x^2 + \frac{\beta'}{\alpha'}x\\
 (x,y)& \longmapsto (\delta^2 x,\ \delta^3 y).
\end{align*}
Or
\[
E_{\lambda'}[4](\F_p)
\simeq
E'[4](\F_p)
\simeq
\mathbb{Z}/2\mathbb{Z} \times \mathbb{Z}/4\mathbb{Z}.
\]
Donc la proposition \ref{carac-legendre} donne
\[
\forall \gamma \in \F_p^\times,
\quad
E_{\lambda'}^{(\gamma)} \text{ est Legendre isomorphe}.
\]
En particulier, $E_{\lambda'}^{(\alpha)}$ est Legendre isomorphe, i.e $E'^{(\alpha)} \simeq E_\mu$.
Enfin, on récapitule 
\[
E^{(\alpha)} \sim E_\lambda \sim E \sim E_{\lambda'},
\qquad
\alpha \notin (\F_p^\times)^2.
\]
D'où
\[
\operatorname{Tr}_{\F_p}(E)
=
-\operatorname{Tr}_{\F_p}(E^{(\alpha)})
=
-\operatorname{Tr}_{\F_p}(E_{\lambda'})
=
+\operatorname{Tr}_{\F_p}(E_{\lambda'}^{(\alpha)})
\]

Donc $E \sim E_{\lambda'}^{(\alpha)} \sim E_\mu$, i.e. \(E\) est Legendre isogène sur \(\F_p\).
\end{enumerate}
\end{itemize}
\[\tag*{$\square$}\]
\end{preuve}


\section{Remarques finales}

Les paragraphes précédents nous ont appris
\[
\begin{cases}
|E_\lambda(\F_p)|= 1 - (-1)^m H_p(\lambda) \mod p\\
|E_\lambda(\F_p)|\in 4\mathbb{Z}.
\end{cases}
\]
\begin{enumerate}
\item En notant $t$ la trace du Frobenius de $E_\lambda/\F_p$, on a $|E_\lambda(\F_p)|= p+1-t$, d’où $t = p+1+4k$.
La borne de Hasse donne $|t|\le 2\sqrt{p}$.

\begin{align*}
|t|\le 2\sqrt{p}
\iff & t^2 \le 4p
\iff (p+1+4k)^2 \le (2\sqrt{p})^2\\
\iff &(p+1+4k+2\sqrt{p})(p+1+4k-2\sqrt{p}) \le 0\\
\iff &(4k+(\sqrt{p}+1)^2)(4k+(\sqrt{p}-1)^2)\le 0\\
\iff & k \in \mathbb{Z} \cap \left[-\frac{(\sqrt{p}+1)^2}{4}\,;\,-\frac{(\sqrt{p}-1)^2}{4}\right]\\
\iff & k \in \left[\mkern-6mu\left[\left\lceil -\frac{(\sqrt{p}+1)^2}{4} \right\rceil \,; \left\lfloor -\frac{(\sqrt{p}-1)^2}{4} \right\rfloor\right]\mkern-6mu\right].
\end{align*}

D’où
\[
|E_\lambda(\F_p)|\in
\left[\mkern-6mu\left[
-4\left\lceil -\frac{(\sqrt{p}-1)^2}{4}\right\rceil\,;\,
-4\left\lfloor -\frac{(\sqrt{p}+1)^2}{4}\right\rfloor
\right]\mkern-6mu\right].
\]
\item De plus les valeurs apparaissant dans $\bigl(H_p(\lambda) \bmod p \,;\, \lambda \in \F_p^2\setminus\{0,1\}\bigr)$ se comprennent grâce à
\[ m \textrm{ est pair }
\iff \frac{p-1}{2} = 2\ell
\iff p = 4\ell+1
\iff p+1-t = 4\ell+2-t
\iff t \equiv 2 \mod 4.
\]

De même $m$ est impair $ \iff t \equiv 0 \mod 4$.
\item 
\[
\bigl(H_p(\lambda) \bmod p \,;\, \lambda \in \F_p^2\setminus\{0,1\}\bigr)
\text{ est un palyndrôme } \iff m \text{ est pair}.
\]

En effet
\[
m \textrm{ est pair } \iff 
(-1) \in (\F_p^\times)^2
\Rightarrow E_\lambda \simeq E_\lambda^{(-1)} \simeq E_{1-\lambda}
\Rightarrow H_p(1-\lambda)=H_p(\lambda).
\]

De même, on a le résultat 
\[
m \textrm{ est impair }
\iff
(-1) \notin (\F_p^\times)^2
\Rightarrow
\operatorname{Tr}_{\F_p}(E_\lambda)
= -\operatorname{Tr}_{\F_p}(E_\lambda^{(-1)})
=- \operatorname{Tr}_{\F_p}(E_{1-\lambda})
\Rightarrow H_p(\lambda) = -H_p(1-\lambda).
\]
\end{enumerate}

\appendix
\section{Couplages de Weil, de Kummer et conséquences}


Soit $m \in \mathbb{N}$, $m \ge 2$, $(m,p)=1$ où $p = \mathrm{car}\,K$.
Soit $E/K$ une courbe elliptique, on a $ E[m] = E[m](\overline{K}) \simeq (\mathbb{Z}/m\mathbb{Z})^2$.

\begin{proposition}
Le couplage de Weil
\[ e_m : E[m] \times E[m] \longrightarrow \mu_m \]
est bilinéaire, alterné, non dégénéré et invariant sous l’action de Galois.
En particulier 
\[
\begin{cases}
\forall S \in E[m],\ e_m(S,T)=1 \Rightarrow T=\mathcal{O}, \\
\forall \sigma \in \Gal(\overline{K}/K),\, e_m(S,T)^\sigma = e_m(S^\sigma, T^\sigma).
\end{cases}
\]
\end{proposition}

\begin{proposition}
Le couplage de Kummer
\begin{align*}
\kappa : E(K) \times \Gal(\overline{K}/K) &\longrightarrow E[m]\\
 (P,\sigma) & \longmapsto Q^\sigma - Q, \qquad \textrm{où }\; Q \in E(\overline{K}) \; /\;  [m]Q = P 
\end{align*}
est bilinéaire, son noyau à gauche est $mE(K)$ et induit
\begin{align*}
\delta_E : E(K)/mE(K) &\longrightarrow \Hom(\Gal(\overline{K}/K), E[m])\\
P &\longmapsto  \delta_E(P) :
\begin{cases}
\Gal(\overline{K}/K) \longrightarrow E[m], \\
\sigma \longmapsto \kappa(P,\sigma).
\end{cases}
\end{align*}
\end{proposition}

\begin{thm}[Hilbert 90]
Supposons que $\mu_m \subset K$.
Alors
\begin{align*}
\delta_K : K^\times/(K^\times)^m &\xrightarrow{\ \sim\ } \Hom(\Gal(\overline{K}/K), \mu_m)\\
b & \longmapsto \delta_K(b) :
\begin{cases}
\Gal(\overline{K}/K) \longrightarrow \mu_m, \\
\sigma \longmapsto \dfrac{\sigma(\beta)}{\beta}, \quad \text{où } \beta^m = b.
\end{cases}
\end{align*}
\end{thm}

\noindent Des trois résultats précédents, on en déduit la construction suivante.
Soient $P \in E(K)$ et $T \in E[m]$, alors
\begin{align*}
\Gal(\overline{K}/K) & \longrightarrow \mu_m\\
\sigma &\longmapsto e_m\bigl(\delta_E(P)(\sigma), T\bigr)
\end{align*}
définit un élément de $\Hom\bigl(\Gal(\overline{K}/K), \mu_m\bigr)$, le théorème 90 de Hilbert assure l’existence d’un unique
$b = b(P,T)$ tel que $ e_m\bigl(\delta_E(P)(\sigma), T\bigr) = \delta_K\bigl(b(P,T)\bigr)(\sigma)$.

\begin{thm}
Il existe une application bilinéaire
\[ b : \frac{E(K)}{mE(K)} \times E[m] \longrightarrow \frac{K^\times}{(K^\times)^m} \]
tel que $b(P,T)$ soit l’unique classe vérifiant
\[ \forall \sigma \in \Gal(\overline{K}/K), \quad e_m\bigl(\delta_E(P)(\sigma), T\bigr) = \delta_K\bigl(b(P,T)\bigr)(\sigma) \]
Le noyau de $b$ à gauche est trivial.
Pour $T \in E[m]$, soient $f_T, g_T \in K(E)$ tels que
\[
\begin{cases}
\mathrm{div}(f_T) = m(T) - m(\mathcal{O}), \\
f_T \circ [m] = g_T^m.
\end{cases}
\]
Alors, pour tout $P \neq T$,
\[b(P,T) \equiv f_T(P) \mod{(K^\times)^m}\]
\end{thm}


\begin{remark}
\begin{enumerate}
\item Si $\Gal(\overline{K}/K)$ agit trivialement sur le $\Gal(\overline{K}/K)$-module $M$, alors
\[
\begin{cases}
H^0\bigl(\Gal(\overline{K}/K), M\bigr) = M, \\
H^1\bigl(\Gal(\overline{K}/K), M\bigr) =  \Hom_{\mathrm{cont}}\bigl(\Gal(\overline{K}/K), M\bigr).
\end{cases}
\]
L’hypothèse $\mu_m \subset K$ de la proposition provient de cette remarque.

\item Notons la mention de morphismes de groupes continus ci-dessus.
Le groupe $\Gal(\overline{K}/K)$ est profini et les groupes $E[m]$, $\mu_m$ sont finis, donc 
\[
\begin{cases}
\Hom\bigl(\Gal(\overline{K}/K), \mu_m\bigr) = \Hom_{\mathrm{cont}}\bigl(\Gal(\overline{K}/K), \mu_m\bigr), \\[0.3em]
\Hom\bigl(\Gal(\overline{K}/K), E[m]\bigr) = \Hom_{\mathrm{cont}}\bigl(\Gal(\overline{K}/K), E[m]\bigr).
\end{cases}
\]
\end{enumerate}
\end{remark}


Revenons à la preuve de la proposition.
Soient $m=2$, $p \ge 3$ premier et
\[
E/\mathbb{F}_p : y^2 = (x-\alpha)(x-\beta)(x-\gamma), \quad \alpha,\beta,\gamma \in \mathbb{F}_p.
\]
Alors $f_{P_\alpha} = x-\alpha \in \mathbb{F}_p(E)$ a pour diviseur $\mathrm{div}(f_{P_\alpha}) = 2(P_\alpha) - 2(O)$, où $P_\alpha = (\alpha,0) \in E(\mathbb{F}_p)$.
De plus, en notant
\[
\begin{cases}
S_1 &= \alpha + \beta + \gamma, \\
S_2 &= \alpha\beta + \alpha\gamma + \beta\gamma,
\end{cases}
\]
on a par un long calcul à partir des règles de calcul de l'addition sur $E$ 
\[ f_{P_\alpha} \circ [2](x,y)
= \left[\frac{x^2 - \alpha x - 2x^2 + 2S_1 x - S_2}{2y}\right]^2
= g_{P_\alpha}^2(x,y).
\]

\noindent Donc, pour tout $Q \neq P_\alpha,\; b(Q,P_\alpha) \equiv f_{P_\alpha}(Q) \mod{(\mathbb{F}_p^\times)^2}$.
De même pour $P_\beta$ et $P_\gamma$.
On a donc la caractérisation
\[Q \in 2E(\mathbb{F}_p) \Leftrightarrow \forall P \in E[2], \quad b(Q,P) \equiv 0\]


Ainsi,
\[
P_\gamma = (\gamma,0) \in 2E(\mathbb{F}_p)
\iff
\begin{cases}
b(P_\gamma,P_\alpha) \equiv 0, \\
b(P_\gamma,P_\beta) \equiv 0, \\
b(P_\gamma,P_\gamma) \equiv 0.
\end{cases}
\]

Or :
\begin{itemize}
\item $b(P_\gamma,P_\alpha) = f_{P_\alpha}(P_\gamma) = \gamma - \alpha$.
\item $b(P_\gamma,P_\beta) = f_{P_\beta}(P_\gamma) = \gamma - \beta$.
\item $b(P_\gamma,P_\gamma) = b(P_\gamma,P_\gamma + P_\alpha - P_\alpha) = b(P_\gamma,P_\gamma + P_\alpha)\, b(P_\gamma ,P_\alpha)^{-1} = b(P_\gamma,P_\beta)\, b(P_\gamma,P_\alpha)^{-1} = \frac{\gamma - \beta}{\gamma - \alpha}$
\end{itemize}

Ainsi
\[P_\gamma = (\gamma,0) \in 2E(\mathbb{F}_p) \iff \gamma - \alpha,\ \gamma - \beta \in (\mathbb{F}_p^\times)^2. \]



\bibliographystyle{alpha}
\bibliography{refs}

\end{document}
