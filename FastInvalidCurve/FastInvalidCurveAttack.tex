%%%%%%%%%%%%%%%%%%%%%%%%%%%%%%%%%
% Header and packages 
%%%%%%%%%%%%%%%%%%%%%%%%%%%%%%%%%

\documentclass[10pt]{article}
\usepackage[utf8]{inputenc}
\usepackage{amsfonts,amssymb,amsmath,amsthm}
\usepackage[a4paper, top=15pt, marginparwidth=25pt, textwidth=510pt, textheight=550pt, bottom=50pt]{geometry}
\usepackage[pdftex]{color,graphicx}
\usepackage{marginnote}


\reversemarginpar
\newtheorem{thm}{Theorem}
\theoremstyle{definition}
\newtheorem{definition}{Definition}
\newtheorem{proposition}{Proposition}

%%%%%%%%%%%%%%%%%%%%%%%%%%%%%%%%%
% Symbols
%%%%%%%%%%%%%%%%%%%%%%%%%%%%%%%%%


\newcommand{\N}{\mathbb{N}}
\newcommand{\F}{\mathbb{F}}
\renewcommand{\L}{\mathbb{L}}
\newcommand{\Z}{\mathbb{Z}}
\newcommand{\Q}{\mathbb{Q}}
\newcommand{\D}{\mathbb{D}}
\newcommand{\R}{\mathbb{R}}
\newcommand{\C}{\mathbb{C}}
\newcommand{\K}{\mathbb{K}}
\renewcommand{\P}{\mathbb{P}}
\newcommand{\Pp}{$\mathcal{P}$}
\newcommand{\ch}{\textrm{ch}}
\newcommand{\sh}{\textrm{sh}}
\newcommand{\id}{\textrm{Id}}
\newcommand{\atan}{\textrm{Arctan}}
\newcommand{\acos}{\textrm{Arccos}}
\newcommand{\asin}{\textrm{Arcsin}}
\newcommand{\Mat}{\textrm{Mat}}
\newcommand{\Gl}{\textup{Gl}}
\newcommand{\Ker}{\textup{Ker}}
\newcommand{\E}{\textrm{E}}
\newcommand{\card}{\textup{card}}
\newcommand{\rg}{\textup{rg}}
\renewcommand{\Im}{\textup{Im}}
\renewcommand{\Gl}{\textrm{Gl}}

\begin{document}

%%%%%%%%%%%%%%%%%%%%%%%%%%%%%%%%%
% Title
%%%%%%%%%%%%%%%%%%%%%%%%%%%%%%%%%

\author{Pierre Chrétien}
\title{Fast Invalid Curve Attack using twists}
\date{February 2025}
\maketitle
\begin{abstract}
Exploiting Invalid Curve Attack is a recurrent problem in CTF. 
We present the common structure of the attack and give some insight to speed up the attack. 
This paper has primarily an expository role.
\end{abstract}

%%%%%%%%%%%%%%%%%%%%%%%%%%%%%%%%%
% Body
%%%%%%%%%%%%%%%%%%%%%%%%%%%%%%%%%

\section{Introduction}

The so called \textsl{Invalid curve attack} is a real threat for cryptographic protocols based on elliptic curves.
The attack has first been presented in cite{BMM00} and the use of twists described in \cite{FLRV08}.
OpenPGP.js prior to 4.2.0 was found to be vulnerable\footnote{https://www.cve.org/CVERecord?id=CVE-2019-9155}. 
Bluetooth has been proven to be vulnerable to a "Fixed Coordinate" variant  \cite{cryptoeprint:2019/1043}.
The SafeCurves website and the associated paper \cite{cryptoeprint:2024/1265} point out as 
\begin{quote}
An ECC implementor can stop an invalid-curve attack by checking whether the input point Q satisfies the correct curve equation; [...]
But this creates a conflict between simplicity and security. An implementation that does not include this check is simpler and more likely to be produced, and will pass typical functionality tests. 
\end{quote}

\noindent It is also at heart of many Capture The Flag and cryptographic challenges on dedicated platforms.


\noindent The rest of the paper is organized as follows.
Section 2 recalls the basics mathematical concepts used  in the sequel, we recall basics facts about discrete logarithm problem (DLP) and twists of elliptic curves.	
Section 3 presents the general setting of the attack and ways to exploit poor implementation and weak curves.

This paper has primarily an expository role.

\section{Background Material}

\textbf{Notations :}
We will denote by $\F_q$ the finite field with $q = p^n$ elements where $p \geq 5$ and $n \in \N - \lbrace 0 \rbrace$.
We will denote by $E/\F_q$ an elliptic curve defined over $\F_q$. 
The reader is assumed to be familiar with basic theory of elliptic curves.

\vspace*{.5cm}

\noindent The characteristic $p$ being different from $2$ and $3$, every elliptic curve $E/\F_q$ may be written as
\[ E : y^2 = x^3 + ax + b, \; \; a,b \in \F_q. \]
This is a so called \textsl{short Weierstrass form} of E.

\vspace*{.5cm}

\noindent \textbf{Remarks :} 
\begin{itemize}
\item The characteristic $p$ being greater than $5$ is not a restriction in our context since $p$ will usually be a large prime number.
\item The short Weierstrass form is not unique.

\begin{verbatim}
k = GF(11**2)
u = k(2)
E = EllipticCurve(k,[1,1])
E_ = EllipticCurve(k,[u**(-4),u**(-6)])
E.is_isomorphic(E_)
E.isomorphism_to(E_)
\end{verbatim}
\end{itemize}

\subsection{Twists of Elliptic Curves}

\begin{itemize}
\item Automorphismes d'une courbe elliptique. 
\item Cas particulier de l'équation short p > 3.
\item Description des twists, expliquer pourquoi j différent de 0 et 1728 n'est pas un problème, donc seulement le twist quadratique à considérer en général, écriture des morphismes
\item exemples
\end{itemize}

\subsection{Discrete Logarithm Problem}

\begin{definition}
Let $G$ be a group in multiplicative notation.
The \textbf{Discrete Logarithm Problem} (DLP) is : given $g,h \in G$ find $a \in \Z$ such that $h = g^a$.
\end{definition}

\noindent \textbf{Remarks :}
\begin{enumerate}
\item Let $\L/\F_q$ be a field extension and $E/\F_q$ an elliptic curve.
The group of $\L$-rationnal points of $E$ denoted $E(\L)$ (that is the solutions $(x,y) \in \L^2$ of the Weierstrass equation defining $E$) is usually written with an additive law. 
So the DLP for elliptic curves may be rephrased : given $P,G \in E(\L)$ find $a \in \Z$ such that $P = aG$.
\end{enumerate}

\textbf{Solving the DLP}
\begin{itemize}
\item Polhig Helman, BSGS, Rho.
\item Sage Discretelog vs Q.log(P)
\end{itemize}


\section{Invalid Curve Attack}
\subsection{General Setting}
Position du problème : les algo de multiplication d*P n'utilisent que le coeff a de E, on peut passer (x,y) n'étant pas sur E.

\begin{itemize}
\item DLP peut etre trop dur sur E à cause de d'un ordre pas assez smooth.
\item Si on peut faire calculer k*T pour T sur une autre courbe E' on peut trouver T modulo les premiers de l'ordre de E'.
\item si le twist a un ordre avec d'autres facteurs premier que E alors on connait k modulo de nouveaux premiers. 
\item cela peut suffire à retrouver k (CRT)
\end{itemize}

\subsection{Exploiting Ladders and twists}
\begin{itemize}
\item L'importance des ladders cf Safe Curves
\item L'importance des multiplications où on ne passe que le x.
\end{itemize}


\bibliographystyle{plain}
\bibliography{refs}

\end{document}
