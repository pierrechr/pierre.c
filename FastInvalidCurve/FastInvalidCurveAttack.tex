%%%%%%%%%%%%%%%%%%%%%%%%%%%%%%%%%
% Header and packages 
%%%%%%%%%%%%%%%%%%%%%%%%%%%%%%%%%

\documentclass[10pt]{article}
\usepackage[utf8]{inputenc}
\usepackage{amsfonts,amssymb,amsmath,amsthm}
\usepackage[a4paper, top=50pt, marginparwidth=25pt, textwidth=510pt, textheight=550pt, bottom=50pt]{geometry}
\usepackage[pdftex]{color,graphicx}
\usepackage{marginnote}


\reversemarginpar
\newtheorem{thm}{Theorem}
\theoremstyle{definition}
\newtheorem{definition}{Definition}
\newtheorem{proposition}{Proposition}
\newtheorem{remark}{Remark}

%%%%%%%%%%%%%%%%%%%%%%%%%%%%%%%%%
% Symbols
%%%%%%%%%%%%%%%%%%%%%%%%%%%%%%%%%


\newcommand{\N}{\mathbb{N}}
\newcommand{\F}{\mathbb{F}}
\renewcommand{\L}{\mathbb{L}}
\newcommand{\Z}{\mathbb{Z}}
\newcommand{\Q}{\mathbb{Q}}
\newcommand{\D}{\mathbb{D}}
\newcommand{\R}{\mathbb{R}}
\newcommand{\C}{\mathbb{C}}
\newcommand{\K}{\mathbb{K}}
\renewcommand{\P}{\mathbb{P}}
\newcommand{\Pp}{$\mathcal{P}$}
\newcommand{\ch}{\textrm{ch}}
\newcommand{\sh}{\textrm{sh}}
\newcommand{\id}{\textrm{Id}}
\newcommand{\atan}{\textrm{Arctan}}
\newcommand{\acos}{\textrm{Arccos}}
\newcommand{\asin}{\textrm{Arcsin}}
\newcommand{\Mat}{\textrm{Mat}}
\newcommand{\Gl}{\textup{Gl}}
\newcommand{\Ker}{\textup{Ker}}
\newcommand{\E}{\textrm{E}}
\newcommand{\card}{\textup{card}}
\newcommand{\rg}{\textup{rg}}
\renewcommand{\Im}{\textup{Im}}
\renewcommand{\Gl}{\textrm{Gl}}

\begin{document}

%%%%%%%%%%%%%%%%%%%%%%%%%%%%%%%%%
% Title
%%%%%%%%%%%%%%%%%%%%%%%%%%%%%%%%%

\author{Pierre Chrétien}
\title{Practical Invalid Curve Attack Using Quadratic Twist}
\date{February 2025}
\maketitle

%%%%%%%%%%%%%%%%%%%%%%%%%%%%%%%%%
% Body
%%%%%%%%%%%%%%%%%%%%%%%%%%%%%%%%%

\section{Introduction}

The so called \textsl{Invalid curve attack} is a real threat for cryptographic protocols based on elliptic curves.
The attack has first been presented in \cite{BMM00} and the use of twists was described in \cite{FLRV08}.
OpenPGP.js prior to 4.2.0 was found to be vulnerable\footnote{https://www.cve.org/CVERecord?id=CVE-2019-9155}. 
The Node.js secp256k1-node allows bindings to the  "Bitcoin curve" \verb|secp256k1| and was found to be vulnerable\footnote{https://nvd.nist.gov/vuln/detail/CVE-2024-48930} to small subgroup attacks.
Bluetooth was proved to be vulnerable to a "Fixed Coordinate" variant  \cite{cryptoeprint:2019/1043}.
Edwards model has also been examined in \cite{NT15}.
The SafeCurves website and the associated paper \cite{cryptoeprint:2024/1265} point out as 
\textsl{"An ECC implementor can stop an invalid-curve attack by checking whether the input point Q satisfies the correct curve equation; [...]
But this creates a conflict between simplicity and security. An implementation that does not include this check is simpler and more likely to be produced, and will pass typical functionality tests."}


The rest of the paper is organized as follows.
Section 2 recalls the basics mathematical concepts used, in particular about discrete logarithm problem (DLP) and twists of elliptic curves. 
Section 3 presents the general setting of the attack and ways to exploit poor implementation and weak curves.
Section 4 is a complete walkthrough example with a toy curve $E$.
A comprehensive study of morphisms and the quadratic twist $E_d$ allows to solve the challenge DLP not in $E(\F_{q^2})$ (as usually seen in Invalid Curve Attack write-ups) but instead in $E(\F_{q})$ and $E_d(\F_{q})$. 
This paper has an expository role.

\section{Background Material}

\textbf{Notations :}
We will denote by $\F_q$ the finite field with $q = p^n$ elements where $p \geq 5$ and $n \in \N - \lbrace 0 \rbrace$.
We will denote by $E/\F_q$ an elliptic curve defined over $\F_q$. 
The reader is assumed to be familiar with basic theory of elliptic curves.


\noindent \textbf{Short Weierstrass equations.}  Since $p\geq 5$, every elliptic curve $E/\F_q$ may be written as
\[ E : y^2 = x^3 + ax + b, \; \; a,b \in \F_q. \]
This is a so called \textsl{short Weierstrass form} of the curve $E$ \textsl{defined over $\F_q$}.
The point at infinity of $E$ is denoted by $\mathcal{O}_E$.

\noindent \begin{remark} 
\begin{enumerate}
\item The condition $p \geq 5$ is not a restriction in our context since $p$ will usually be a large prime.
\item A short Weierstrass form is not unique.
This will be completed in the subsection about twists.
\end{enumerate}
\end{remark}

\noindent \textbf{Automorphisms.} Let $E_1/\F_q$ and $E_2/\F_q$ be elliptic curves. 
These curves may be seen over $\overline{\F_q}$ that is, the coefficients of their equation may be seen as lying in $\overline{\F_q}$ instead of in $\F_q$.
Every geometric isomorphism of elliptic curve $\phi$ from $E_1/\overline{\F_q}$ to $E_2/\overline{\F_q}$ has an  affine part of the form 
\begin{equation}\label{morphism}
 \phi(x,y) = (u^{\mkern1mu 2}x + r, u^{\mkern1mu 3}y + su^{\mkern1mu 2}x + t).
\end{equation}
for $u \in \overline{\F_q}^{\mkern1mu *}$, $r,s,t \in \overline{\F_q}$.
We will denote geometric isomorphism as $\phi/\overline{\F_q}$.
The isomorphism $\psi$ is said to be \textsl{defined over $\F_q$} or \textsl{rational} if $u,r,s,t \in \F_q$, we will denote it by $\psi/\F_q$.

\begin{proposition}\label{prop:auto}
Let $E_i/\F_q$, $i \in \lbrace 1; 2 \rbrace$ be elliptic curves given by short Weierstrass equations.
\[ E_i : y^2 = x^3 + a_ix + b_i, \; \; a_i,b_i \in \F_q. \]
A geometric isomorphism $\phi$ froom $E_1$ to $E_2$ is of the form 
\[ \phi(x,y) = (u^{\mkern1mu 2}x, u^{\mkern1mu 3}y).\]  
\end{proposition}
\begin{proof}
This is included as a first step to fully understand isomorphisms in the quadratic twist case.

\noindent Let $(x,y) \in E_1$ and $\phi$ as given by (\ref{morphism}).
Applying $\phi$ to the equation of $E_1$ and expanding yields 
\begin{align*}
&y^2 = x^3 + a_1x + b_1\\
 & \Leftrightarrow ( u^{\mkern1mu 3}y + su^{\mkern1mu 2}x + t)^{\mkern1mu 2} =  (u^{\mkern1mu 2}x +r)^{\mkern1mu 3} +a_1(u^{\mkern1mu 2}x +r) +b_1 \\
& \Leftrightarrow  u^{\mkern1mu 6}y^2 + s^2u^{\mkern1mu 4}x^2 + t^2 + 2u^{\mkern1mu 5}sxy + 2u^{\mkern1mu 3}ty + 2tsu^{\mkern1mu 2}x  
=  u^{\mkern1mu 6}x^3 + 3ru^{\mkern1mu 4}x^2 + 3r^2 u^{\mkern1mu 2}x +r^{\mkern1mu 3} +a_1u^{\mkern1mu 2}x +a_1r +b_1 (*)\\
\end{align*} 
Identifying coefficients of $xy$ and $y$ with those of $y^2 = x^3 + a_2x + b_2$ yields $s=0, t= 0$ (recall that $u \neq 0$ and $ p \neq 2$).
\begin{align*}
(*) & \Leftrightarrow  u^{\mkern1mu 6}y^2   
=  u^{\mkern1mu 6}x^3 + 3ru^{\mkern1mu 4}x^2 + 3r^2 u^{\mkern1mu 2}x +r^{\mkern1mu 3} +a_1u^{\mkern1mu 2}x +a_1r +b_1 
\end{align*}
Then, identifying the coefficient of $x^2$ with the short equation of $E_2$ yields $r=0$ (here we use $p \neq 3$).
Thus $\phi(x,y) = (u^2x,u^3y)$.
We conclude with the following computations that will be used in the sequel.
\begin{align*}
 &u^{\mkern1mu 6}y^2   
=  u^{\mkern1mu 6}x^3 +a_1u^{\mkern1mu 2}x  +b_1 \\
& \Leftrightarrow  y^2   
=  x^3 +\frac{a_1}{u^{\mkern1mu 4}}x  + \frac{b_1}{u^6} \\
& \Leftrightarrow
\frac{a_1}{u^{\mkern1mu 4}} = a_2 , \; \;  \frac{b_1}{u^6}=b_2 (**)
\end{align*}
\end{proof}


\begin{proposition}
Let $E/\F_q$ be an elliptic curves given by short Weierstrass equation $ E : y^2 = x^3 + ax + b, \; \; a, b \in \F_q$.
The \textsl{$j$-invariant of E} is defined to be 
\[j(E) = 1728 \frac{4a^3}{4a^3+27b^2}.\]

\noindent Let $E_1/\F_q, E_2/\F_q$ be elliptic curves, there exists an isomorphism $\phi / \overline{\F_q}$ from $E_1$ to $E_2$ if and only if $j(E_1) = j(E_2)$. 
\end{proposition}

\noindent \begin{remark}
\begin{enumerate}
\item We insist that the $j$-invariant classifies \textbf{geometric} isomorphism classes of elliptic curves over $\F_q$. 
\item Thanks to $p \geq 5$, $E$ has a short equation and $j(E)$ a special form in this case.
Thus $j(E) \notin \lbrace 0, 1728 \rbrace \Leftrightarrow a,b \in \F_q^{*}$.
\end{enumerate}
\end{remark}

\subsection{Twists of Elliptic Curves}

\noindent \textbf{Twists.} Non trivial twists of $E/\F_q$ are elliptic curves $E'/\F_q$ being isomorphic to $E$ when viewed over $\overline{\F_q}$ but not isomorphic to $E$ when viewed over $\F_q$.
Let $E/\F_q$ be an elliptic curve in short Weierstrass equation $y^2 = x^3+ax+b$.
Recall that $q = p^n$ and $p \geq 5$, so it is possible to write such an equation for $E$. 

\begin{definition}
Le $E/\F_q$ be an elliptic curve.
A \textsl{twist of $E$} is an elliptic curve $E_t/\F_q$ such that there is a geometric isomorphism $\phi/\overline{\F_q}$ of elliptic curves $\phi : E \simeq E_t$.
A twist $E_t$ of $E$ is \textsl{trivial} if there exists an isomorphism $\psi$ of elliptic curve defined over $\F_q$.
\end{definition}

\begin{definition}
Let $d \in \F_q^*$.
The \textsl{twist $E_d$ of $E$ by d} is the elliptic curve given in short Weierstrass equation
\[ E_d : y^2 = x^3 + d^{\mkern1mu 2}ax + d^{\mkern1mu 3}b.\]
\end{definition}

\begin{remark}
We did not specify that $E_d$ is a non trivial.
Actually, let $\delta$ be a square root of $d$ in $\overline{\F_q}$ i.e. $\delta^2 = d$, then
\begin{align*}
\phi : E & \to E_d\\
      (x,y)& \mapsto \big(\frac{x}{d},\frac{y}{d\delta}\big)
\end{align*}
is a geometric isomorphism from $E$ to $E_d$.
It matches the relations (**) concluding proof of Proposition \ref{prop:auto} with $a_1=a, b_1=b, a_2 = ad^2, b_2=bd^3$ and $d = \frac1u$.
\end{remark}

\begin{proposition}\label{prop:trivialtwist}
Assume that $j(E) \neq 0,1728$.
The twist $E_d$ is trivial if and only if $d \in (\F_q^*)^2$.
\end{proposition}

\begin{proof}
$(\Rightarrow)$ Assume that there exists a rational isomorphism $\psi$ from $E$ to $E_d$.
According to Proposition \ref{prop:auto}
\[ \exists u \in \F_q^*, \mkern4mu \psi(x,y) = (u^2x,u^3y) \]

According to (**), $\frac{a}{u^4} = ad^2$ and $\frac{b}{u^6} = bd^3$.
Recall that since $ p \geq 5$, the assumption about $j(E)$ is equivalent to $a,b \neq 0$.
Thus $\frac{1}{u^4} = d^2$, $\frac{1}{u^6} = d^3$ and $d = \frac{d^3}{d^2} = \frac{1}{u^2} \in (\F_q^*)^2$.

$(\Leftarrow)$ Conversely, let $\delta \in \F_q^*$ such that $\delta^2 = d$.
Then  a rational isomorphism from $E$ to $E_d$ is
\[ \psi(x,y) = \big(\frac{x}{d},\frac{y}{d\delta}\big) \]
\end{proof}


\begin{proposition}
Assume that $E/\F_q$ has $j(E) \neq 0, 1728$. 
Then a twist $E_t/\F_q$ of $E/\F_q$ is either trivial or $E_d$ for some $d \in (\F_q^*) \backslash (\F_q^*)^2$.
\end{proposition}
\begin{proof}
Assume that $E_t/\F_q$ is a non trivial twist of $E/\F_q$ with isomorphism $\phi : E \to E_t$ given by $\phi(x,y) = (u^2x,u^3y)$, $u \in \overline{\F_q}$, $u \notin \F_q$.
Let $E_t : y^2 = x^3 +a_tx+b_t$, $a_t,b_t \in \F_q$, thus (**) yields
\[ a_t = \frac{a}{u^4}, b_t = \frac{b}{u^6} \]
Then $u^2 = \frac{ba_t}{ab_t} \in \F_q$, i.e. $u \notin \F_q$ but $u^2 \in \F_q$.
This means that $u \in \F_{q^2} \backslash \F_q$.
Let $d := \frac{1}{u^2}$, then $a_t = d^2a, b_t = d^3b$ and $\phi(x,y) = \big(\frac{x}{d},\frac{uy}{d} \big)$.
\end{proof}


\noindent \begin{remark}
Proposition \ref{prop:trivialtwist} is wrong if $p < 5$.
For example, let $p=q=3$, $E : y^2 = x^3 + x + 1$ and $d = 2$.
In this case, $E$ is isomorphic to $E_d : y^2 = x^3+x+2$  with $\psi$ given by constants $(u,r,s,t) = (2,2,0,0)$.
So $E_d$ is a trivial twist of $E$ but $d$ is non square modulo $3$.
Note that the definition for the $j$-invariant when $p=3$ gives here $j(E) = 0$.
\end{remark}

\noindent \textbf{Order of group of rational points.} The group of rational points $E(\F_q)$ has order $\sharp E(\F_q) = q + 1 -t$ where $t$ is the \textsl{Trace of Frobenius}.
An extensive description of the Frobenius endomorphism is out scope for this paper, we only need some basics facts we recall below.

\begin{proposition}
\begin{enumerate}
\item \textbf{(Hasse Bound)} One has $|t|  \leq 2 \sqrt{q}$.
\item One has $\sharp{} E_d(\F_q) = q+1+t$, thus $|\sharp{} E_d(\F_q) - \sharp{}E(\F_q)| = |2t|  \leq 4\sqrt{q}$.
\end{enumerate}
\end{proposition}
\begin{proof}
\begin{enumerate}
\item The proof of Hasse Bound is technical, the interessted reader may refer to V Theorem 1.1 from \cite{Silverman:EC} 
\item We include this proof to give some insight into how points distribute over $E(\F_q)$, $E_d(\F_q)$.
We will prove that 
\[  \sharp{}E(\F_q) + \sharp{}E_d(\F_q) = 2q+2 \mkern20mu (\dagger)\]
substituting $\sharp E(\F_q) = q + 1 -t$ in $(\dagger)$ gives the result.

Let $f(x) = x^3+ax+b$, $d \in \F_q^*$ be non square in $\F_q$ and $E_d : Y^2 = X^3+d^2aX+d^3b$.
The following change of variables on $E_d$ (take care, this is not an automorphism of $E_d$ but a tool to count points)
\[ X = dx, \mkern2mu Y = dy \]
yields $ Y^2 = X^3+d^2aX+d^3b \Leftrightarrow d^2y^2 = d^3x^3+d^3ax+d^3b \Leftrightarrow y^2 = df(x)$. 

\begin{itemize}
\item Let $x \in \F_q$ such that $f(x) = 0$, then $(x,0) \in E(\F_q)$ and $(dx,0) \in E_d(\F_q)$.
Each curve gets one point.
\item Let $x \in \F_q$ such that $f(x) \in (\F_q^*)^2$.
Then $y^2 = f(x)$ has two solutions.
Since $d$ is non square, $df(x)$ is non square and $y^2 = df(x)$ has no solution in $\F_q$.
So $E$ gets two points and $E_d$ zero.
\item  Let $x \in \F_q$ such that $f(x) \notin (\F_q^*)^2$.
Then $df(x) \in (\F_q^*)^2$ and $y^2 = df(x)$ has two solution giving rise to two points on $E_d$.
Here, $E$ gets zero point.
\end{itemize}
Let $E_a(\F_q) = E(\F_q) \backslash \lbrace \mathcal{O}_E \rbrace$ be the affine part of $E$ and $E_{d,a}(\F_q) = E_d(\F_q) \backslash \lbrace \mathcal{O}_{E_d} \rbrace$ be the affine part of $E_d$.
Each $x \in \F_q$ contributes for $2$ points in the affine part of $E_a(\F_q) \cup E_{d,a}(\F_q)$, note that $E_a(\F_q) \cap E_{a,d}(\F_q) =\emptyset $.
Counting points at infinity once for each curve yields $(\dagger)$.
\end{enumerate}
\end{proof}

\begin{remark}
It is very significant that every $x \in \F_q$ lifts to $E(\F_{q^2})$ but about an half lifts to $E(\F_q)$.
Tracking the coefficient $d$ as in the proof allows to lift points either in $E(\F_q)$ or in $E_d(\F_q)$, that is avoiding any reference to $\F_{q^2}$.
The ultimate goal of this section about twist is this remark.
It allows to solve the DLP not in $E(\F_{q^2})$ but in $E(\F_q)$ or $E_d(\F_q)$, speeding up the attack.
\end{remark}

\subsection{Discrete Logarithm Problem}

\begin{definition}
Let $G$ be a group in multiplicative notation.
The \textbf{Discrete Logarithm Problem} (DLP) is : given $b,h \in G$ find $a \in \Z$ such that $h = b^a$.
\end{definition}

\noindent \begin{remark}
\begin{enumerate}
\item The group law on an elliptic curve being usually written in additive notation, DLP for elliptic curves is rephrased as : given $P,B \in E(\F_q)$ find $a \in \Z$ such that $P = aB$.
\item The DLP has a solution if and only if $P$ is in the subgroup \textless$B$\textgreater~generated by $B$.
\end{enumerate}
\end{remark}

The following discussion gives the necessaries notions when dealing with generic methods to solve the DLP that will be used in the last section.
The generic \verb|discrete_log| method from Sage uses a combination of Pohlig-Hellman, Baby Step Giant Step (BSGS) , Pollard’s kangaroo (i.e. Pollard's Lambda), and Pollard’s Rho. 

\begin{itemize}
\item The Pohlig-Helmann method.
Let $n$ be the order of a point $B \in E(\F_q)$ and  $n = \prod_{i = 1}^m p_i^{n_i}$ be its prime factorization.
The subgroup $H$ generated by $B$ is cyclic of order $n$, thus has a unique cyclic subgroup $H_i$ of order $p_i^{n_i}$ for each $i \in [| 1; m|]$.
By means of the Chinese Remainder Theorem (CRT), solving the DLP in $H$ boils down to solve it in each $H_i$. 
The subgroups $H_i$ have order a prime power, which may still be quite large. 
One can reduce the DLP from $H_i$ to subgroups $\tilde{H}_i$ of order \textbf{exactly} $p_i$.
We restrict ourselves to solving the DLP in the $\tilde{H}_i$'s.
\item BSGS method is a collision finding algorithm to solve DLP that requires $\mathcal{O}(\sqrt{p_i})$ running time and $\mathcal{O}(\sqrt{p_i})$ storage.
This gives an bound on the size of the $p_i$'s for which we can hope to solve the DLP with this method.
\item Pollard's Rho (resp. Lambda) algorithm is solving the DLP and has $\mathcal{O}(\sqrt{p_i})$ (resp. $\mathcal{O}(\sqrt{p_i})$) time complexity but $\mathcal{O}(1)$ ) (resp. $\mathcal{O}(\ln{p_i})$) space complexity.
\end{itemize}

\section{Invalid Curve Attack}

Invalid Cuve Attack is presented in \cite{FLRV08}.
Let $E/\F_q : y^2 = x^3+ax+b$ be the public curve of the cryptosystem on which DLP is (assumed) hard.

\vspace*{.5cm}
\textbf{Main idea :} The vulnerabilty lies in the fact that a malicious user may send a point that is \textbf{not} on $E/\F_q$ but on a weaker curve, say a quadratic twist, and then exploit it to solve an easier DLP.
This is mainly due to $b$ not beeing used for scalar multiplication on $E$ in short Weierstrass form.
Let's break it down.

\begin{enumerate}
\item Mallory may send an honest $P = (x,y)$ on $E$ and gets back $B:=$\verb|k|$.P$ from Bob.
Mallory wants to recover the secret key \verb|k|.
But Mallory may also send a malicious point $Q = ( \tilde{x},\tilde{y})$ 
on another curve $\tilde{E} : y^2 = x^3+ax+\tilde{b}$ (note that the $a$ coefficient remains unchanged).
Since formulae for computing \verb|k|$.Q$ do not involve $b, \tilde{b}$, Bob will correctly compute \verb|k|$.Q \in \tilde{E}$, believing he just made a computation on $E$.
\item Varying the curve $\tilde{E}$,  i.e. varying $\tilde{b} \in \F_q$, Mallory collects various \verb|k|$.Q_i$ for $Q_i \in \tilde{E}_i$.
Choosing wisely the $\tilde{E}_i$'s and $Q_i$'s may produce several easy DLP.
\item Say that Mallory chooses $\tilde{E}_i$ and  $Q_i \in \tilde{E}_i$ of order $n_i$, let $p_i$ be a prime divisor of $n_i$.
Sending $Q_i$ to Bob, Mallory receives $B_i:=$\verb|k|$.Q_i$.
Note that $\hat{Q}_i=\frac{n_i}{p_i}Q_i$ has order $p_i$.
Mallory computes $\hat{B}_i=\frac{n_i}{p_i}B_i$, so that $\hat{B}_i =  \verb|k|.\hat{Q}_i$.
Mallory solves this DLP in the group \textless$\hat{Q}_i$\textgreater~ of order $p_i$, thus recovers \verb|k| modulo $p_i$. 
\item Iterating this process over various prime factors $p_i$ of $n_i$, for various $n_i$ on selected $\tilde{E}_i$ gives \verb|k| modulo many primes.
By means of the CRT, Mallory recovers \verb|k| modulo \textbf{the product} of the $p_i$'s.
If \verb|k| is known to be a, say, $2^{256}$ bits key and the product of the $p_i$ is greater than $2^{256}$ then Mallory actually recovered  \verb|k|.
\end{enumerate}

\begin{remark}
\begin{enumerate}
\item Checking that the received point $Q$ acutally lies on $E$ stops the attack.
\item According to \cite{cryptoeprint:2024/1265}, a curve supporting \textsl{"simple, fast, constant-time single-coordinate single-scalar multiplication [...]drastically limits the power of invalid-curve attacks"} due to the fact that for a given $x \in \F_q$, formulae for single-coordinate ladders work for the original curve and the quadratic twist.
So if the implementation is based on such ladders (e.g. Montgomery ladders), Mallory has only access to points on $E$ and $E_d$ but on no other curve $\tilde{E}$ as above.
\item Even if $\sharp{} E(\F_q) = hp$ with $p$ prime and $h$ a small cofactor (a classical setting in cryptographic applications) then $\sharp{}E_d(\F_q)$ might have a prime factorization with many small primes (such numbers are called \textsl{smooth}).
For example \textbf{brainpoolP256t1} curve has prime order but its quadratic twist has smooth order since it factors in a product of $7$ primes of which $6$ have bit length less than $42$ (the last prime factor has bit length $89$).
\end{enumerate}
\end{remark}

\section{Walkthrough Example}

In this section we give full details to implement the attack on a toy curve in a fake game between Bob and Mallory.
The full code is available in \verb|FastInvalidAttack.py|, containing some more explanations and comments.

\vspace*{.5cm}
\textbf{Setting :} 
\begin{itemize}
\item You are given some ECDH challenge, assume moreover that the curve $E$ is in short Weierstrass form.
\item The implementation on Bob's side does not check whether the user passes a point actually on $E$.
\item If the challenge uses two coordinates to compute on $E$, one may try a Small Subgroup Attack.
So one may choose $x$ such that it lifts as $Q_i=(x,y)$ of smooth order on a weak curve $E_i$.
This is not the situation we are interested in.
\item If the challenge uses single-coordinate to compute on $E$ (with some sort of Montgomery scalar multiplication) this may be an Invalid Curve Attack restricting the choice of the weak $E_i$ to the quadratic twist $E_d$.
\end{itemize}

\noindent Assume we are given a challenge where one plays the role of a malicious user Mallory interacting with Bob.
Mallory sends a point $P$ to Bob who computes $B := $ \verb|k|$.P$ for some secret key \verb|k| and sends back $B$.
Mallory goal is to recover the secret key \verb|k|.
Assume moreover that Bob uses some sort of single-coordinate scalar multiplication.

\begin{enumerate}
\item Check the prime factorizations of $\sharp{}E(\F_q), \sharp{}E_d(\F_q)$.
Check that they are smooth enough, i.e. have many prime factors with bit length less than, say, $40$.
Each of these orders only have one very large prime factor.
Moreover check that these orders have \textbf{different} small prime factors.
This implies that sending to Bob a point on $E_d$ allows Mallory to recover the secret key \verb|k| modulo some new prime factors compared to only sending points that lie on $E$.
\item One chooses a point $G_0 \in \E_d(\F_q)$ by lifting some $x\in \F_q$, this is done by trial and error until one finds $x$ such that $f(x) \notin (\F_q)^2$. 
\item Since $\sharp{}E_d(\F_q)$ has only one unhandleable big prime factor $p_b$ for DLP, we restrict to the subgroup of order $n_0 = \frac{\sharp{}E_d(\F_q)}{p_b}$ by setting $\hat{G}_0 : = p_b G_0$.
\item One chooses a point $G_1 \in \E(\F_q)$ by lifting some $x\in \F_q$, this is done by trial and error until one finds $x$ such that $f(x) \in (\F_q)^2$. 
\item Since $\sharp{}E(\F_q)$ has only one unhandleable big prime factor $p_b'$ for DLP, we restrict to the subgroup of order $n_1 = \frac{\sharp{}E(\F_q)}{p_b'}$ by setting $\hat{G}_1 : = p_b' G_1$.
\item Mallory sends the $x$-coordinates of $\hat{G}_0$ and $\hat{G}_1$ to Bob and gets back \verb|pubkey0| and \verb|pubkey1|.
Those are $x$-coordinates of points $P_0 \in E_d(\F_q)$ and $P_1 \in E(\F_q)$ such that 
\[ P_i = \verb|k|\hat{G}_i \]
This is where using morphisms between $E$ and $E_d$ allows Mallory to dodge working in $\F_{q^2}$.
Note that Bob uses a single-coordinate scalar multiplication (and no extra bit to identify $y$), thus Mallory knows the discrete logs only up to sign. 
Indeed, $P_i$ and $-P_i$ have the same $x$-coordinate due to the short Weierstrass form, Mallory may have recovered \verb|k| or -\verb|k| (modulo $n_i$).
\item Mallory solves these two DLP to recover \verb|k| modulo $n_0$  and $n_1$.
By means of the CRT, Mallory recovers \verb|k| modulo $n_0n_1$ (note that $n_0, n_1$ are coprime).
Since $n_0 \times n_1 \geq \sharp{}E(\F_q)$, Mallory  actually recovered \verb|k|.


\end{enumerate}

\begin{remark}
Note that points 2. and 3. above are usually done in Invalid Curve Attack write-ups in $E_d(\F_{q^2})$, thus the DLP involving $G_0$ is solved over $\F_{q^2}$ implying a significant slow down of the attack.
If the challenge has to be solved under some time constraints, this could be an hindering.
\end{remark}


\bibliographystyle{plain}
\bibliography{refs}


\end{document}
